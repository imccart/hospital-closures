\documentclass[a4paper, 12pt]{article}
\usepackage{amsmath, amsfonts, amssymb}
\usepackage{array}
\usepackage{booktabs}
\usepackage[figurename=Figure]{caption}
\usepackage{float}
\usepackage[margin=2.3cm]{geometry}
\usepackage{graphicx}
\usepackage{hyperref}
\usepackage[utf8]{inputenc}
\usepackage{multirow}
\usepackage{setspace}
\usepackage{subcaption}
\usepackage{pdflscape}

\hypersetup{
  colorlinks=false,
  linkcolor=black,
  filecolor=black,
  urlcolor=black,
  citecolor=black,
}

\setlength{\tabcolsep}{3pt}

\newcolumntype{L}[1]{>{\raggedright\let\newline\\\arraybackslash\hspace{0pt}}m{#1}}
\newcolumntype{C}[1]{>{\centering\let\newline\\\arraybackslash\hspace{0pt}}m{#1}}

\raggedbottom

\begin{document}
\onehalfspacing

\begin{center}
{\LARGE Supplemental Appendix} \\[0.5cm]
{\large Survival versus Scale: The Effects of Critical Access Hospital Designation}
\end{center}

\vspace{1cm}

This appendix provides supplementary material for the main text. Section~\ref{sec:app-financial} details the construction of our hospital-level financial measures. Section~\ref{sec:app-robustness} presents robustness checks varying the lag structure for state-level CAH adoption in the control group definition. Section~\ref{sec:app-diagnostics} reports diagnostics for the synthetic difference-in-differences (SDID) estimator, including sample attrition from balanced-panel requirements, pre-trend tests, and synthetic control weight concentration. Section~\ref{sec:app-preperiod} examines the sensitivity of financial outcome estimates to the choice of pre-period length. Section~\ref{sec:app-cs} reports Callaway--Sant'Anna event-study estimates and results from the eligibility-restricted control group design. Section~\ref{sec:app-fect} presents interactive fixed effects estimates. Section~\ref{sec:app-perm} reports permutation inference results for the SDID estimates. Section~\ref{sec:app-anticipation} examines the sensitivity of capacity estimates to anticipatory downsizing.


%%% ===================================================================
%%% SECTION A: Financial Measure Construction
%%% ===================================================================

\section{Financial Measure Construction}
\label{sec:app-financial}

We construct six financial measures---operating margin, current ratio, net patient revenue per bed, total operating expenses per bed, net fixed assets per bed, and capital expenditures per bed---by combining data from the Healthcare Cost Report Information System (HCRIS) and IRS Form 990 filings. HCRIS is the primary source for all measures; Form 990 serves as a secondary source when HCRIS data are unavailable for a given hospital-year. This section describes the construction of each measure, the source hierarchy, and the post-processing steps applied to the combined series.

\subsection{Data Sources}

\paragraph{\textit{HCRIS.}} Medicare-certified hospitals file annual cost reports, from which we extract net patient revenue, total operating expenses, current assets, current liabilities, net fixed assets (land, buildings, and equipment net of accumulated depreciation), and the components of accumulated depreciation. These fields are drawn from Worksheet G (balance sheet) and Worksheet G-3 (income statement) of the cost report. When a hospital files multiple reports in a single fiscal year, we sum the relevant fields across reports.

\paragraph{\textit{Form 990.}} Nonprofit hospitals file IRS Form 990, from which we extract total revenue, total expenses, total assets, total liabilities, net fixed assets, and depreciation expense. Each AHA hospital identifier may link to up to three Employer Identification Numbers (EINs) through our fuzzy matching procedure (see the data documentation). When multiple EINs are available, we use the first non-missing value in EIN priority order.

\subsection{Measure Definitions}

\paragraph{\textit{Operating margin.}} From HCRIS, operating margin is defined as
$$
\text{margin}_{\text{HCRIS}} = \frac{\text{net patient revenue} - \text{total operating expenses}}{\text{net patient revenue}}.
$$
When HCRIS data are unavailable, we substitute the Form 990 analogue,
$$
\text{margin}_{990} = \frac{\text{total revenue} - \text{total expenses}}{\text{total revenue}},
$$
restricted to observations with $\text{margin}_{990} \in (-1, 1)$. The two definitions differ in scope: the HCRIS measure reflects patient-care operations, while the Form 990 measure includes all organizational revenue and expenses.

\paragraph{\textit{Current ratio.}} From HCRIS, the current ratio is defined as
$$
\text{current ratio}_{\text{HCRIS}} = \frac{\text{current assets}}{\text{current liabilities}},
$$
where both fields are drawn from the balance sheet (Worksheet G). When HCRIS data are unavailable, we substitute the Form 990 analogue, which uses total assets divided by total liabilities. Because the Form 990 version includes long-term items, it will tend to exceed the HCRIS current ratio for the same hospital; the HCRIS measure is therefore preferred when available.

\paragraph{\textit{Net fixed assets per bed.}} Net fixed assets are taken directly from the HCRIS balance sheet when available, with Form 990 net fixed assets as a fallback. We normalize by each hospital's pre-period mean bed count, computed as the average of total staffed beds (AHA variable \texttt{BDTOT}) across all years after 1995 and prior to CAH designation (or across all post-1995 years for hospitals that never receive designation). We deflate to constant 2010 dollars using the annual CPI-U and express the result in millions of dollars per bed:
$$
\text{net fixed assets per bed} = \frac{\text{net fixed assets}}{10^6 \times \bar{B}_{\text{pre}} \times \text{CPI deflator}},
$$
where $\bar{B}_{\text{pre}}$ is the pre-period mean bed count and the CPI deflator is the ratio of the current-year CPI-U index to the 2010 index.

\paragraph{\textit{Capital expenditures per bed.}} We approximate capital expenditures as the year-to-year change in gross fixed assets. From HCRIS, gross fixed assets are computed as net fixed assets plus accumulated depreciation; capital expenditure is then the first difference of this series within each hospital. When HCRIS data are unavailable, we use the Form 990 analogue: the change in net fixed assets plus depreciation expense. As with net fixed assets, we normalize by the pre-period mean bed count and deflate to constant 2010 dollars, expressed in units of \$10,000 per bed:
$$
\text{capex per bed} = \frac{\Delta\text{gross fixed assets}}{10^4 \times \bar{B}_{\text{pre}} \times \text{CPI deflator}}.
$$

\paragraph{\textit{Net patient revenue per bed.}} Net patient revenue is taken from the HCRIS income statement when available. When HCRIS data are unavailable, we substitute total revenue from the Form 990. We normalize by each hospital's pre-period mean bed count and deflate to constant 2010 dollars using the CPI-U, expressed in thousands of dollars per bed:
$$
\text{net patient revenue per bed} = \frac{\text{net patient revenue}}{10^3 \times \bar{B}_{\text{pre}} \times \text{CPI deflator}}.
$$

\paragraph{\textit{Total operating expenses per bed.}} Total operating expenses are taken from the HCRIS income statement when available, with total expenses from the Form 990 as a fallback. As with revenue, we normalize by the pre-period mean bed count and deflate to constant 2010 dollars, expressed in thousands of dollars per bed:
$$
\text{operating expenses per bed} = \frac{\text{total operating expenses}}{10^3 \times \bar{B}_{\text{pre}} \times \text{CPI deflator}}.
$$

\noindent Because operating margin is defined as $(R - C)/R$, these two component measures allow us to assess whether changes in margins reflect movements in revenue, costs, or both. The revenue and expense measures are constructed after computing operating margin from raw values, so the winsorization and per-bed normalization applied to these components do not affect the margin calculation.

\subsection{Post-Processing}

\paragraph{\textit{Winsorization.}} We winsorize all six financial measures at the 5th and 95th percentiles within each year-by-ever-CAH cell. Winsorization is applied after combining HCRIS and Form 990 sources but before CPI deflation and per-bed normalization of the capital and revenue/expense measures. This reduces the influence of extreme values driven by reporting anomalies while preserving cross-sectional and temporal variation within plausible ranges.

\paragraph{\textit{Interpolation.}} After winsorization, we linearly interpolate missing values within each hospital's time series using observed year as the time index. Interpolation fills isolated gaps (e.g., a single missing cost report year) while leaving leading and trailing missingness unchanged. The same interpolation procedure is applied to the capacity and staffing variables (total beds, OB beds, FTE RNs, and inpatient days per bed).


%%% ===================================================================
%%% SECTION B: Robustness to Control Group Lag
%%% ===================================================================

\section{Robustness to Varying State-Level CAH Adoption Lags}
\label{sec:app-robustness}

Our main specification constructs the not-yet-treated control group by requiring that a state's first CAH designation occur strictly after the cohort year. Formally, for a treated cohort converting in year $i$, a not-yet-treated hospital enters the control pool only if its state's first CAH designation year satisfies $s > i + \delta$, where $\delta = 0$ in the baseline. Increasing $\delta$ strengthens the exclusion by removing states whose CAH programs begin shortly after the cohort year, reducing the risk that anticipation effects or early program spillovers contaminate the control group. The cost is a smaller control pool, which reduces precision and may alter the composition of comparison units.

Figure~\ref{fig:statecut-sensitivity} presents SDID estimates for all outcomes at $\delta = 1$ and $\delta = 2$ alongside the baseline ($\delta = 0$). We do not report $\delta = 3$ because the earliest cohort (1999) can only draw controls from states adopting in 2003 or later, leaving too few comparison units for reliable estimation. The shaded band in each panel spans the baseline 95\% confidence interval, and the dashed vertical line marks the baseline point estimate. Across all outcomes, the sensitivity estimates remain qualitatively similar to the baseline: capacity reductions, system membership increases, and null margin effects persist at both lag values. Point estimates generally fall within or near the baseline confidence band, confirming that our findings are not driven by contamination from states that adopt CAH programs shortly after the cohort year.

\begin{figure}[H]
\centering
\includegraphics[width=\textwidth, height=0.8\textheight, keepaspectratio]{../results/statecut-sensitivity.png}
\caption{Robustness of SDID estimates to varying state-level CAH adoption lags ($\delta$). Each panel displays one outcome. The shaded band spans the baseline ($\delta = 0$) 95\% confidence interval; the dashed line marks the baseline point estimate. Points show SDID ATTs at $\delta = 1$ and $\delta = 2$, with 95\% confidence intervals. Circles and squares denote $\delta = 1$ and $\delta = 2$, respectively. For hospital-level outcomes, cohort-lag combinations with fewer than 5 control hospitals in the balanced panel are excluded from estimation; missing points indicate that no cohort met this threshold at the given lag.}
\label{fig:statecut-sensitivity}
\end{figure}

We also examine sensitivity to the bed size cutoff used to define the sample of plausibly CAH-eligible hospitals. Our baseline restricts to hospitals with 50 or fewer beds, providing a buffer above the statutory 25-bed limit to account for hospitals that downsize to qualify. Figure~\ref{fig:bedcut-sensitivity} reports SDID estimates using cutoffs of 25 and 75 beds. Tightening the cutoff to 25 beds restricts the sample to hospitals already at or below the statutory limit, while relaxing it to 75 beds includes larger hospitals that may be less comparable to CAH converters. State-level outcomes (closures and mergers) are excluded because they do not depend on the hospital-level bed size filter. Across all hospital-level outcomes, the estimates are broadly consistent with the baseline, confirming that the results are not artifacts of the particular bed size threshold.

\begin{figure}[H]
\centering
\includegraphics[width=\textwidth, height=0.85\textheight, keepaspectratio]{../results/bedcut-sensitivity.png}
\caption{Robustness of SDID estimates to varying bed size cutoffs. Each panel displays one hospital-level outcome. The shaded band spans the baseline (50-bed cutoff) 95\% confidence interval; the dashed line marks the baseline point estimate. Points show SDID ATTs at 25-bed and 75-bed cutoffs, with 95\% confidence intervals. Cohort-cutoff combinations with fewer than 5 control hospitals in the balanced panel are excluded.}
\label{fig:bedcut-sensitivity}
\end{figure}


%%% ===================================================================
%%% SECTION C: SDID Diagnostics
%%% ===================================================================

\section{SDID Diagnostics}
\label{sec:app-diagnostics}

Our primary estimates use the synthetic difference-in-differences (SDID) estimator, which requires a balanced panel of units observed in every period of the event window. This requirement can reduce sample sizes relative to estimators that accommodate unbalanced panels, such as Callaway and Sant'Anna (CS). Similarly, the SDID estimator assigns nonnegative weights to control units to match the treated group's pre-period trajectory; if these weights are highly concentrated on a small number of controls, the resulting estimates may be sensitive to idiosyncratic variation in those units. We present three diagnostics below to assess these concerns.

%%% TABLE 1: Sample Comparison
\begin{landscape}
\enlargethispage{3cm}
\begin{table}[H]
\centering
\caption[Sample Comparison: SDID Balanced Panel vs.\ Callaway--Sant'Anna]{Sample Comparison: SDID Balanced Panel vs.\ Callaway--Sant'Anna. ``Treated'' and ``Control'' report the number of units (hospitals for Panels A--B; states for Panel C) in each estimator's sample. ``\% Lost'' is the percentage of units dropped from the SDID sample when balancing the panel. Hospital-level outcomes filter to hospitals with at most 50 beds in all observed years.}
\label{tab:diag-sample}
\footnotesize
\renewcommand{\arraystretch}{0.85}
\input{../results/diagnostics/diag-sample-comparison.tex}
\end{table}
\end{landscape}

%%% TABLE 2: Pre-Trend Tests
\begin{table}[H]
\centering
\caption{Pre-Trend Tests: Interaction of Event Time and Treatment in Pre-Periods}
\label{tab:diag-pretrend}
\small
\input{../results/diagnostics/diag-pretrend.tex}
\begin{minipage}{0.85\linewidth}
\vspace{0.3cm}
\footnotesize
\textit{Notes:} Each row reports the coefficient on the interaction of event time and treatment status, estimated on pre-treatment observations only. Standard errors are clustered at the hospital level for hospital-level outcomes and at the state level for state-level outcomes. $^{*}$~$p<0.10$, $^{**}$~$p<0.05$, $^{***}$~$p<0.01$.
\end{minipage}
\end{table}

The FTE RN outcome shows evidence of differential pre-trends in some cohorts. This is consistent with treated hospitals---small, rural facilities pursuing CAH designation---already experiencing staffing declines before the policy change. Estimates for FTE RNs should be interpreted with this caveat in mind.

The revenue and expense decomposition outcomes also exhibit differential pre-trends. Total operating expenses per bed shows statistically significant trends in all three cohorts ($p < 0.002$ in each case), while net patient revenue per bed flags in the 1999 cohort ($p = 0.003$) and marginally in 2001 ($p = 0.06$). However, as shown in Section~\ref{sec:app-preperiod}, SDID point estimates for these outcomes are stable across pre-period lengths of 2--5 years, suggesting the estimates are not mechanically driven by differential trends. We therefore present the revenue/expense decomposition as suggestive evidence rather than drawing strong causal conclusions from these outcomes alone.

%%% PRE-TREND FIGURES
\clearpage

Figures~\ref{fig:pretrend-financial}--\ref{fig:pretrend-org} plot mean outcomes for treated and control groups by event time, averaged across cohorts. The vertical dashed line marks the last pre-treatment period.

\begin{figure}[H]
\centering
\begin{subfigure}[b]{0.48\textwidth}
  \includegraphics[width=\textwidth]{../results/diagnostics/diag-pretrend-margin.png}
\end{subfigure}
\hfill
\begin{subfigure}[b]{0.48\textwidth}
  \includegraphics[width=\textwidth]{../results/diagnostics/diag-pretrend-current_ratio.png}
\end{subfigure}
\\[0.5cm]
\begin{subfigure}[b]{0.48\textwidth}
  \includegraphics[width=\textwidth]{../results/diagnostics/diag-pretrend-net_fixed.png}
\end{subfigure}
\hfill
\begin{subfigure}[b]{0.48\textwidth}
  \includegraphics[width=\textwidth]{../results/diagnostics/diag-pretrend-capex.png}
\end{subfigure}
\\[0.5cm]
\begin{subfigure}[b]{0.48\textwidth}
  \includegraphics[width=\textwidth]{../results/diagnostics/diag-pretrend-net_pat_rev.png}
\end{subfigure}
\hfill
\begin{subfigure}[b]{0.48\textwidth}
  \includegraphics[width=\textwidth]{../results/diagnostics/diag-pretrend-tot_operating_exp.png}
\end{subfigure}
\caption{Pre/post trends: Financial performance outcomes}
\label{fig:pretrend-financial}
\end{figure}

\begin{figure}[H]
\centering
\begin{subfigure}[b]{0.48\textwidth}
  \includegraphics[width=\textwidth]{../results/diagnostics/diag-pretrend-bdtot.png}
\end{subfigure}
\hfill
\begin{subfigure}[b]{0.48\textwidth}
  \includegraphics[width=\textwidth]{../results/diagnostics/diag-pretrend-obbd.png}
\end{subfigure}
\\[0.5cm]
\begin{subfigure}[b]{0.48\textwidth}
  \includegraphics[width=\textwidth]{../results/diagnostics/diag-pretrend-ftern.png}
\end{subfigure}
\hfill
\begin{subfigure}[b]{0.48\textwidth}
  \includegraphics[width=\textwidth]{../results/diagnostics/diag-pretrend-ip_per_bed.png}
\end{subfigure}
\caption{Pre/post trends: Capacity and staffing outcomes}
\label{fig:pretrend-capacity}
\end{figure}

\begin{figure}[H]
\centering
\begin{subfigure}[b]{0.48\textwidth}
  \includegraphics[width=\textwidth]{../results/diagnostics/diag-pretrend-system.png}
\end{subfigure}
\hfill
\begin{subfigure}[b]{0.48\textwidth}
  \includegraphics[width=\textwidth]{../results/diagnostics/diag-pretrend-closures.png}
\end{subfigure}
\\[0.5cm]
\begin{subfigure}[b]{0.48\textwidth}
  \includegraphics[width=\textwidth]{../results/diagnostics/diag-pretrend-mergers.png}
\end{subfigure}
\caption{Pre/post trends: Organizational change outcomes}
\label{fig:pretrend-org}
\end{figure}

%%% TABLE 3: Weight Concentration
\clearpage

\begin{table}[H]
\centering
\caption{SDID Synthetic Control Weight Concentration}
\label{tab:diag-weights}
\small
\input{../results/diagnostics/diag-weights.tex}
\begin{minipage}{0.85\linewidth}
\vspace{0.3cm}
\footnotesize
\textit{Notes:} ``$N$ Controls'' is the number of control units in the balanced panel. ``Max Weight'' is the largest single omega weight assigned by the SDID estimator. ``Top 5 Share'' is the sum of the five largest weights. ``HHI'' is the Herfindahl--Hirschman Index of the omega weights ($\sum \omega_i^2$), where values closer to zero indicate more dispersed weighting. Hospital-level outcomes use individual hospitals as units; state-level outcomes (closures, mergers) use states.
\end{minipage}
\end{table}

%%% ===================================================================
%%% SECTION D: Pre-Period Sensitivity for Financial Outcomes
%%% ===================================================================

\section{Pre-Period Sensitivity for Financial Outcomes}
\label{sec:app-preperiod}

Financial outcomes rely on data from HCRIS and IRS Form 990 filings, which have sparser pre-treatment coverage than the AHA survey data used for capacity and staffing outcomes. In particular, HCRIS coverage ramps up gradually through the late 1990s, so that requiring longer pre-periods mechanically reduces the balanced panel. To assess whether our SDID estimates for financial outcomes are sensitive to the choice of pre-period length, we re-estimate each financial outcome using pre-periods of 2, 3, 4, and 5 years. The baseline specification in the main text uses 5 pre-periods.

Table~\ref{tab:preperiod-financial} reports the results. Operating margin estimates are consistently near zero across all pre-period lengths ($-0.010$ to $0.004$), confirming that the null margin result is not an artifact of any particular pre-period choice. Net fixed assets per bed is similarly stable, ranging from $-0.099$ to $-0.072$. Current ratio estimates attenuate from $-0.709$ (pre-period~$=5$) to $-0.418$ (pre-period~$=2$), losing statistical significance at the shortest pre-period, but remaining negative throughout.

The decomposition outcomes---net patient revenue per bed and total operating expenses per bed---are remarkably stable across pre-period lengths. Revenue estimates range from $-200$ to $-219$ thousand 2010 dollars per bed, and expense estimates range from $-177$ to $-208$. The near-identical co-movement of revenue and expenses persists at every pre-period length, reinforcing the proportional contraction interpretation. Capital expenditures per bed remain uninformative at all pre-period lengths.

As the pre-period shortens, the treated sample grows from 34 hospitals (pre-period~$=5$) to 43 (pre-period~$=2$), and the control pool expands from 139 to 319. The stability of point estimates across this range of sample sizes provides additional reassurance that the results are not driven by a particular subset of hospitals that happen to have unusually long financial histories.

\begin{landscape}
\begin{table}[H]
\centering
\caption{Pre-Period Sensitivity: Financial Outcome SDID Estimates}
\label{tab:preperiod-financial}
\small
\begin{tabular}{l cc cc cc cc}
\toprule
 & \multicolumn{2}{c}{Pre $= 5$} & \multicolumn{2}{c}{Pre $= 4$} & \multicolumn{2}{c}{Pre $= 3$} & \multicolumn{2}{c}{Pre $= 2$} \\
\cmidrule(lr){2-3} \cmidrule(lr){4-5} \cmidrule(lr){6-7} \cmidrule(lr){8-9}
Outcome & ATT & 95\% CI & ATT & 95\% CI & ATT & 95\% CI & ATT & 95\% CI \\
\midrule
Operating margin & -0.010 & [-0.05, 0.02] & 0.000 & [-0.04, 0.04] & -0.005 & [-0.04, 0.03] & 0.004 & [-0.03, 0.04] \\
Current ratio & -0.709 & [-1.31, -0.11] & -0.745 & [-1.27, -0.22] & -0.604 & [-1.11, -0.09] & -0.418 & [-0.86, 0.02] \\
Net fixed assets & -0.099 & [-0.27, 0.07] & -0.093 & [-0.25, 0.06] & -0.075 & [-0.23, 0.08] & -0.072 & [-0.22, 0.07] \\
Capital expenditures per bed & -1.246 & [-5.8, 3.3] & -2.655 & [-8.4, 3.0] & -0.476 & [-5.3, 4.3] & -0.255 & [-4.92, 4.41] \\
Net patient revenue per bed & -203.2 & [-583, 177] & -218.7 & [-622, 185] & -212.3 & [-597, 173] & -200.3 & [-549, 149] \\
Operating expenses per bed & -207.6 & [-623, 207] & -193.6 & [-565, 178] & -184.7 & [-539, 169] & -177.2 & [-507, 153] \\
\midrule
Control obs. & \multicolumn{2}{c}{139} & \multicolumn{2}{c}{191} & \multicolumn{2}{c}{263} & \multicolumn{2}{c}{319} \\
Treated obs. & \multicolumn{2}{c}{34} & \multicolumn{2}{c}{38} & \multicolumn{2}{c}{39} & \multicolumn{2}{c}{43} \\
\bottomrule
\end{tabular}

\begin{minipage}{0.95\linewidth}
\vspace{0.3cm}
\footnotesize
\textit{Notes:} Each cell reports the pooled SDID ATT and 95\% confidence interval for the indicated financial outcome and pre-period length. Revenue and expense measures are in thousands of 2010 dollars per bed. The pre-period length determines the number of years before CAH designation required to enter the balanced panel; shorter pre-periods admit more hospitals. ``Control obs.'' and ``Treated obs.'' report the number of hospitals in the SDID balanced panel at each pre-period length.
\end{minipage}
\end{table}
\end{landscape}

%%% ===================================================================
%%% SECTION E: Callaway–Sant'Anna and Eligibility-Restricted Results
%%% ===================================================================

\section{Callaway--Sant'Anna Event-Study Estimates and Alternative Control Group Results}
\label{sec:app-cs}

This section presents two sets of supplementary results. First, we report Callaway and Sant'Anna (CS) event-study estimates for all outcomes, complementing the SDID estimates in the main text. The CS estimator accommodates unbalanced panels and traces dynamic treatment effects over event time, a useful complement to SDID when balanced-panel requirements induce sample attrition. Second, we report ATT estimates from the eligibility-restricted control group design in Section~4.6 of the main text.

\subsection{CS Overall Results: State-Timing Design}

Table~\ref{tab:cs-overall} reports CS ATT estimates for all outcomes under the state-timing control group construction used in the main text. These estimates use the ``not-yet-treated'' control group specification and are aggregated across cohorts using the CS group-time ATT framework.

\begin{table}[H]
\centering
\footnotesize
\caption{Callaway--Sant'Anna ATT Estimates: State-Timing Design}
\label{tab:cs-overall}
\begin{tabular}{lcl}
Outcome & CS ATT & CS 95\% CI \\
Operating margin & 0.001 & [-0.04, 0.04] \\
Current ratio & 1.816 & [1.30, 2.33] \\
Net fixed assets & -0.154 & [-0.29, -0.01] \\
Capital expenditures per bed & 0.810 & [-2.75, 4.37] \\
Net patient revenue per bed & -522.009 & [-709.83, -334.18] \\
Operating expenses per bed & -539.866 & [-728.77, -350.96] \\
Total beds & -16.779 & [-30.84, -2.72] \\
OB beds & -4.280 & [-6.10, -2.46] \\
FTE RNs & -60.022 & [-72.70, -47.35] \\
Inpatient days per bed & -83.638 & [-117.85, -49.42] \\
System membership & 0.145 & [0.02, 0.27] \\
Closures & -0.295 & [-0.60, 0.01] \\
Mergers & -0.096 & [-0.55, 0.36] \\
\end{tabular}

\begin{minipage}{0.85\linewidth}
\vspace{0.3cm}
\footnotesize
\textit{Notes:} ATT estimates from the Callaway--Sant'Anna estimator with ``not-yet-treated'' control group specification. Revenue and expense measures are in thousands of 2010 dollars per bed. Closures and mergers are state-level counts per 100 hospitals.
\end{minipage}
\end{table}

\subsection{CS Event-Study Figures: State-Timing Design}

Figures~\ref{fig:cs-financial}--\ref{fig:cs-org} present CS event-study estimates for each outcome, plotting dynamic treatment effects over event time relative to CAH designation.

\begin{figure}[H]
\centering
\begin{subfigure}[b]{0.48\textwidth}
  \includegraphics[width=\textwidth]{../results/margin-cs.png}
\end{subfigure}
\hfill
\begin{subfigure}[b]{0.48\textwidth}
  \includegraphics[width=\textwidth]{../results/currentratio-cs.png}
\end{subfigure}
\\[0.5cm]
\begin{subfigure}[b]{0.48\textwidth}
  \includegraphics[width=\textwidth]{../results/netpatrev-cs.png}
\end{subfigure}
\hfill
\begin{subfigure}[b]{0.48\textwidth}
  \includegraphics[width=\textwidth]{../results/totopexp-cs.png}
\end{subfigure}
\\[0.5cm]
\begin{subfigure}[b]{0.48\textwidth}
  \includegraphics[width=\textwidth]{../results/netfixed-cs.png}
\end{subfigure}
\hfill
\begin{subfigure}[b]{0.48\textwidth}
  \includegraphics[width=\textwidth]{../results/capex-cs.png}
\end{subfigure}
\caption{CS event-study estimates: Financial performance and capital outcomes}
\label{fig:cs-financial}
\end{figure}

\begin{figure}[H]
\centering
\begin{subfigure}[b]{0.48\textwidth}
  \includegraphics[width=\textwidth]{../results/beds-cs.png}
\end{subfigure}
\hfill
\begin{subfigure}[b]{0.48\textwidth}
  \includegraphics[width=\textwidth]{../results/beds_ob-cs.png}
\end{subfigure}
\\[0.5cm]
\begin{subfigure}[b]{0.48\textwidth}
  \includegraphics[width=\textwidth]{../results/ftern-cs.png}
\end{subfigure}
\hfill
\begin{subfigure}[b]{0.48\textwidth}
  \includegraphics[width=\textwidth]{../results/ipdays-cs.png}
\end{subfigure}
\caption{CS event-study estimates: Capacity and staffing outcomes}
\label{fig:cs-capacity}
\end{figure}

\begin{figure}[H]
\centering
\begin{subfigure}[b]{0.48\textwidth}
  \includegraphics[width=\textwidth]{../results/closure-rate-cs.png}
\end{subfigure}
\hfill
\begin{subfigure}[b]{0.48\textwidth}
  \includegraphics[width=\textwidth]{../results/merger-rate-cs.png}
\end{subfigure}
\\[0.5cm]
\begin{subfigure}[b]{0.48\textwidth}
  \includegraphics[width=\textwidth]{../results/system-cs.png}
\end{subfigure}
\caption{CS event-study estimates: Organizational change outcomes}
\label{fig:cs-org}
\end{figure}


\subsection{Eligibility-Restricted Control Group Results}

Table~\ref{tab:elig-overall} reports overall ATT estimates from the eligibility-restricted design, which uses all non-CAH or not-yet-CAH hospitals with $\leq 50$ beds as controls regardless of state-level CAH adoption timing. This design expands the cohort range to 1999--2005 and substantially increases treated sample sizes relative to the state-timing design. Closures and mergers are omitted because these outcomes are estimated at the state level and require the state-timing design.

\begin{table}[H]
\centering
\footnotesize
\caption{ATT Estimates: Eligibility-Restricted Control Group}
\label{tab:elig-overall}
\begin{tabular}{lccrcc}
\toprule
Outcome & SDID ATT & SDID 95\% CI & $N_{tr}$ & CS ATT & CS 95\% CI \\
\midrule
Operating margin & -0.012 & [-0.02, -0.00] & 219 & -0.011 & [-0.03, 0.01] \\
Current ratio & -0.121 & [-0.30, 0.05] & 219 & 0.500 & [0.21, 0.79] \\
Net fixed assets & -0.037 & [-0.07, -0.00] & 218 & -0.107 & [-0.16, -0.05] \\
Capital expenditures per bed & -0.997 & [-2.01, 0.02] & 190 & -1.809 & [-3.23, -0.39] \\
Net patient revenue per bed & -2.477 & [-60.9, 56.0] & 219 & -234.6 & [-323, -146] \\
Operating expenses per bed & 13.35 & [-45.0, 71.7] & 220 & -206.7 & [-289, -124] \\
\addlinespace
Total beds & -6.69 & [-8.2, -5.2] & 875 & -0.720 & [-5.2, 3.7] \\
OB beds & -0.652 & [-0.79, -0.51] & 860 & -0.483 & [-0.82, -0.15] \\
FTE RNs & -3.163 & [-4.12, -2.20] & 875 & -2.670 & [-6.3, 1.0] \\
Inpatient days per bed & -17.60 & [-21.2, -14.0] & 875 & 4.015 & [-4.4, 12.5] \\
\addlinespace
System membership & 0.048 & [0.02, 0.08] & 287 & 0.050 & [-0.01, 0.11] \\
\bottomrule
\end{tabular}

\begin{minipage}{0.85\linewidth}
\vspace{0.3cm}
\footnotesize
\textit{Notes:} ATT estimates from SDID and Callaway--Sant'Anna estimators using the eligibility-restricted control group (all non-CAH hospitals with $\leq 50$ beds). Cohorts span 1999--2005. $N_{tr}$ reports the total number of treated units across cohorts in the SDID balanced panel. Revenue and expense measures are in thousands of 2010 dollars per bed.
\end{minipage}
\end{table}

\clearpage

\subsection{CS Event-Study Figures: Eligibility-Restricted Design}

Figures~\ref{fig:elig-cs-financial}--\ref{fig:elig-cs-capacity} present CS event-study estimates from the eligibility-restricted design.

\begin{figure}[H]
\centering
\begin{subfigure}[b]{0.48\textwidth}
  \includegraphics[width=\textwidth]{../results/elig-margin-cs.png}
\end{subfigure}
\hfill
\begin{subfigure}[b]{0.48\textwidth}
  \includegraphics[width=\textwidth]{../results/elig-currentratio-cs.png}
\end{subfigure}
\\[0.5cm]
\begin{subfigure}[b]{0.48\textwidth}
  \includegraphics[width=\textwidth]{../results/elig-netpatrev-cs.png}
\end{subfigure}
\hfill
\begin{subfigure}[b]{0.48\textwidth}
  \includegraphics[width=\textwidth]{../results/elig-totopexp-cs.png}
\end{subfigure}
\\[0.5cm]
\begin{subfigure}[b]{0.48\textwidth}
  \includegraphics[width=\textwidth]{../results/elig-netfixed-cs.png}
\end{subfigure}
\hfill
\begin{subfigure}[b]{0.48\textwidth}
  \includegraphics[width=\textwidth]{../results/elig-capex-cs.png}
\end{subfigure}
\caption{CS event-study estimates (eligibility-restricted): Financial performance and capital outcomes}
\label{fig:elig-cs-financial}
\end{figure}

\begin{figure}[H]
\centering
\begin{subfigure}[b]{0.48\textwidth}
  \includegraphics[width=\textwidth]{../results/elig-beds-cs.png}
\end{subfigure}
\hfill
\begin{subfigure}[b]{0.48\textwidth}
  \includegraphics[width=\textwidth]{../results/elig-beds_ob-cs.png}
\end{subfigure}
\\[0.5cm]
\begin{subfigure}[b]{0.48\textwidth}
  \includegraphics[width=\textwidth]{../results/elig-ftern-cs.png}
\end{subfigure}
\hfill
\begin{subfigure}[b]{0.48\textwidth}
  \includegraphics[width=\textwidth]{../results/elig-ipdays-cs.png}
\end{subfigure}
\\[0.5cm]
\begin{subfigure}[b]{0.48\textwidth}
  \includegraphics[width=\textwidth]{../results/elig-system-cs.png}
\end{subfigure}
\caption{CS event-study estimates (eligibility-restricted): Capacity, staffing, and system membership}
\label{fig:elig-cs-capacity}
\end{figure}


%%% ===================================================================
%%% SECTION F: Interactive Fixed Effects Estimates
%%% ===================================================================

\section{Interactive Fixed Effects Estimates}
\label{sec:app-fect}

Our primary SDID estimates require balanced panels, which can substantially reduce sample sizes---particularly for financial outcomes with sparse pre-treatment data coverage. As a robustness check, we estimate treatment effects using the interactive fixed effects (IFE) counterfactual estimator implemented in the \texttt{fect} package (Liu, Wang, and Xu, 2024). The IFE estimator models untreated potential outcomes as a function of unit and time fixed effects plus a low-rank factor structure, then imputes counterfactual outcomes for treated observations. Crucially, this approach accommodates unbalanced panels natively, allowing us to retain the full sample of hospitals with available data.

We apply the IFE estimator to all financial and operational outcomes using the eligibility-restricted control group construction (cohorts 1999--2005). Financial outcomes use a five-year pre-period matching the SDID specification; operational outcomes (beds, OB beds, FTE RNs, inpatient days per bed, and system membership) also use a five-year pre-period. For each cohort, we select the number of factors (0--5) by leave-one-out cross-validation and compute bootstrap standard errors (150 replications). Cohort-level ATTs are aggregated using the same treated-count weighting as our SDID estimates.

\begin{table}[H]
\centering
\footnotesize
\caption{Interactive Fixed Effects ATT Estimates}
\label{tab:gsynth}
\begin{tabular}{lccr}
\toprule
Outcome & IFE ATT & 95\% CI & $N_{tr}$ \\
\midrule
Operating margin & -0.021 & [-0.03, -0.01] & 668 \\
Current ratio & -0.257 & [-0.46, -0.06] & 829 \\
Net fixed assets & -0.052 & [-0.11, 0.01] & 816 \\
Capital expenditures per bed & -0.547 & [-1.54, 0.44] & 656 \\
Net patient revenue per bed & -107.7 & [-368, 152] & 430 \\
Operating expenses per bed & -202.3 & [-329, -76] & 753 \\
\addlinespace
Total beds & -6.41 & [-8.0, -4.8] & 753 \\
OB beds & -0.577 & [-0.76, -0.40] & 679 \\
FTE RNs & -4.394 & [-5.9, -2.9] & 764 \\
Inpatient days per bed & -23.27 & [-28.5, -18.0] & 455 \\
\addlinespace
System membership & 0.034 & [0.00, 0.07] & 571 \\
\bottomrule
\end{tabular}

\begin{minipage}{0.85\linewidth}
\vspace{0.3cm}
\footnotesize
\textit{Notes:} ATT estimates from the interactive fixed effects (IFE) counterfactual estimator. The number of factors is selected by cross-validation for each cohort. Revenue and expense measures are in thousands of 2010 dollars per bed. $N_{tr}$ reports the total number of treated units across cohorts. Bootstrap standard errors based on 150 replications.
\end{minipage}
\end{table}


%%% ===================================================================
%%% SECTION G: Permutation Inference
%%% ===================================================================

\section{Permutation Inference}
\label{sec:app-perm}

The SDID estimator computes standard errors via the jackknife, which may perform poorly with small numbers of treated units. As a nonparametric alternative, we implement a permutation test. For each outcome, we randomly reassign treatment status among units within each cohort's balanced panel, holding the number of treated units fixed, and re-estimate the SDID ATT. We repeat this procedure 500 times to construct a placebo distribution. The permutation $p$-value is the fraction of placebo ATTs at least as extreme in absolute value as the actual estimate.

Figure~\ref{fig:permutation} displays the placebo distributions for all outcomes. The dashed red line marks the actual SDID estimate, and the annotation reports the two-sided permutation $p$-value.

\begin{figure}[H]
\centering
\includegraphics[width=\textwidth]{../results/permutation-sdid.png}
\caption{Permutation Inference for SDID Estimates}
\label{fig:permutation}
\begin{minipage}{0.85\linewidth}
\vspace{0.3cm}
\footnotesize
\textit{Notes:} Each panel shows the distribution of 500 placebo SDID ATTs obtained by randomly reassigning treatment labels within each cohort's balanced panel. The dashed red line marks the actual SDID estimate. The $p$-value is the two-sided share of placebo ATTs at least as large in absolute value as the observed ATT.
\end{minipage}
\end{figure}


\section{Anticipatory Downsizing}
\label{sec:app-anticipation}

As documented in the main text, hospitals begin reducing beds and utilization in advance of formal CAH designation, consistent with the eligibility requirements imposing real operational constraints that hospitals must satisfy before conversion. If hospitals actively restructure in the year immediately preceding designation, then the standard event-study assumption that treatment begins at $t=0$ may attribute some of the treatment effect to the pre-period, attenuating estimated effects and potentially generating apparent pre-trend violations in capacity outcomes.

To assess the sensitivity of our results to this anticipatory behavior, we re-estimate the SDID specification from the eligibility-restricted design (Section~4.6 of the main text) after redefining treatment as beginning one year before formal CAH designation (i.e., at $t=-1$ rather than $t=0$). This shifts the last pre-treatment period from $t=-1$ to $t=-2$, excluding the year of most active pre-designation adjustment from the donor-matching window. Table~\ref{tab:anticipation} reports the results.

\begin{table}[H]
\centering
\caption{SDID Estimates: Baseline vs.\ Anticipation-Adjusted Treatment Timing}
\label{tab:anticipation}
\begin{tabular}[t]{lcccc}
\toprule
 & \multicolumn{2}{c}{Baseline ($t = 0$)} & \multicolumn{2}{c}{Anticipation ($t = -1$)} \\
\cmidrule(lr){2-3} \cmidrule(lr){4-5}
Outcome & ATT & 95\% CI & ATT & 95\% CI \\
\midrule
Operating margin & -0.012 & [-0.02, -0.00] & -0.020 & [-0.03, -0.01] \\
Current ratio & -0.121 & [-0.30, 0.05] & -0.196 & [-0.36, -0.03] \\
Net fixed assets & -0.037 & [-0.07, -0.00] & -0.057 & [-0.10, -0.01] \\
Capital expenditures per bed & -0.997 & [-2.01, 0.02] & -1.219 & [-2.27, -0.17] \\
Net patient revenue per bed & -2.477 & [-60.9, 56.0] & -82.13 & [-169, 5] \\
Operating expenses per bed & 13.35 & [-45.0, 71.7] & -66.29 & [-148, 15] \\
\addlinespace
Total beds & -6.69 & [-8.2, -5.2] & -6.17 & [-7.6, -4.8] \\
OB beds & -0.652 & [-0.79, -0.51] & -0.628 & [-0.77, -0.49] \\
FTE RNs & -3.163 & [-4.12, -2.20] & -3.656 & [-4.54, -2.77] \\
Inpatient days per bed & -17.60 & [-21.2, -14.0] & -20.35 & [-24.0, -16.7] \\
\addlinespace
System membership & 0.048 & [0.02, 0.08] & 0.049 & [0.02, 0.08] \\
\bottomrule
\end{tabular}

\begin{minipage}{0.85\linewidth}
\vspace{0.3cm}
\footnotesize
\textit{Notes:} Both columns report pooled SDID ATTs from the eligibility-restricted design (cohorts 1999--2005). The baseline column defines treatment at the year of formal CAH designation ($t=0$). The anticipation column redefines treatment as starting one year earlier ($t=-1$), so that pre-period matching uses only $t \leq -2$. Confidence intervals are based on jackknife standard errors.
\end{minipage}
\end{table}


\end{document}
