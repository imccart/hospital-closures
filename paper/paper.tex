\documentclass[a4paper, 12pt, twoside]{article}
\usepackage{amsmath, amsfonts, amscd, epsfig, amssymb, amsthm}
\usepackage{array}%Columns in table
\usepackage{arydshln}
\usepackage[english]{babel}
\usepackage{booktabs}%Horizontal lines in tables
\usepackage[figurename=Figure]{caption}
\usepackage{float}
\usepackage[margin=2.3cm]{geometry}
%\usepackage{graphicx}
%\usepackage{placeins}
\usepackage{hyperref}
\usepackage[utf8]{inputenc}
\usepackage{multirow}
\usepackage{multicol}
\usepackage{setspace}
\usepackage{subcaption}
\usepackage{tocloft}
\usepackage{url}
\usepackage{pdflscape}
\usepackage{comment}
\usepackage{tikz}
\usetikzlibrary{arrows,shapes,positioning,shadows,trees}

\makeatletter
\renewcommand\paragraph{\@startsection{paragraph}{4}{\z@}%
   {1.25ex \@plus 1ex \@minus .2ex}%
   {-1em}%
   {\normalfont\itshape}}
\makeatother

\usepackage[round]{natbib}
\bibliographystyle{aer}

%Avoid random spacing
\raggedbottom

%Set space between columns
\setlength{\tabcolsep}{3pt}

%Define column in tables
\newcolumntype{L}[1]{>{\raggedright\let\newline\\\arraybackslash\hspace{0pt}}m{#1}}
\newcolumntype{C}[1]{>{\centering\let\newline\\\arraybackslash\hspace{0pt}}m{#1}}
\newcolumntype{R}[1]{f>{\raggedleft\let\newline\\\arraybackslash\hspace{0pt}}m{#1}}

%Define colours
\hypersetup{
	colorlinks=false,
	linkcolor=black,
	filecolor=black,      
	urlcolor=black,
	citecolor=black,
}

%Define superindices
\newcommand{\ts}{\textsuperscript}

%Define dashed lines REMOVE
\setlength\dashlinedash{10.0pt}
\setlength\dashlinegap{1.5pt}
\setlength\arrayrulewidth{0.3pt}


%%%%%%%%%%%%%%%%%%%%%%%%%%%%%%%%%%%%%%%%%%%%%
%%%%%%%%%%%%%%%%%%%%%%%%%%%%%%%%%%%%%%%%%%%%%

\begin{document}
\pagenumbering{gobble}
\thispagestyle{empty}

\title{
\vspace{-1.5cm}\Huge
Survival versus Scale: The Effects of Critical Access Hospital Designation
%\thanks{} 
}


\author{
\large Katie Leinenbach\footnote{\scriptsize Demand Side Analytics. Email: katie.leinenbach@gmail.com} \hspace{0.2cm} \large Ian M. McCarthy\footnote{\scriptsize Department of Economics, Emory University, and NBER. Email: ian.mccarthy@emory.edu}}
\date{February 2026}
\maketitle

\onehalfspace

\begin{abstract}
\noindent Hospitals are critical inputs into patient health and, increasingly, local economies. And yet policy-makers continue to struggle with the increasing rate of hospital closures, particularly in rural areas. Created by the Balanced Budget Act in 1997, Critical Access Hospital (CAH) designation was designed to offer financial assistance to hospitals in financial distress, focusing on hospitals in rural areas. We explore how CAH designation affects a hospital's financial position, its effects on hospital scope, and ultimately its effects on hospital closures and mergers.

%\textbf{Keywords:} 
\end{abstract}
\clearpage

\pagenumbering{arabic}
\setcounter{page}{2}


\section{Introduction} 
\label{sec:introduction}
Access to hospital care is central to maintaining quality of life and economic well-being, particularly in rural and underserved communities \citep{ziedan2022}. Beyond their clinical role, hospitals increasingly serve as anchors of local labor markets and economic activity, with the broader healthcare sector now representing the nation’s largest employer \citep{alexander2023}. Despite this centrality, the United States has experienced a steady rise in hospital closures over recent decades, with closures disproportionately concentrated among small and rural facilities \citep{medpac2021}. These trends have intensified policy concern that financially fragile hospitals may be unable to sustain operations in low-density markets, motivating a range of federal interventions designed to stabilize hospital finances and preserve access to care.

This paper asks whether designation-based hospital subsidies preserve meaningful access to care, or whether they primarily preserve hospital presence at a smaller operational scale. We focus on the Critical Access Hospital (CAH) program, the largest and longest-running federal designation aimed at stabilizing financially distressed hospitals. Conceptually, CAH designation creates an inherent tradeoff: by increasing reimbursement, it may reduce financial distress and prevent closure, but by conditioning eligibility on strict limits to inpatient volume and service scope, it may also constrain hospitals’ capacity. Whether CAH designation “helps” therefore depends on which of these margins dominates—preserving hospital survival or preserving the scale at which care is delivered.

The CAH program provides a canonical example of such designation-based support. Officially established under the Balanced Budget Act of 1997, the program offers eligible hospitals cost-based reimbursement for Medicare patients, effectively guaranteeing a positive margin on a large share of inpatient revenue. In exchange, hospitals must satisfy ongoing eligibility requirements tied to size, location, and service provision, including limits on inpatient census, length of stay, and scope of services. While cost-based reimbursement may reduce short-run financial risk, these institutional features together shape hospitals' behavioral responses to designation. In practice, the accompanying eligibility constraints may induce substantial operational restructuring that offsets the reimbursement benefit, so that designation need not translate into an improvement in hospitals' overall financial position.

Credibly estimating the effects of CAH designation is complicated by staggered policy adoption and endogenous hospital take-up. States adopted the CAH program at different points between 1999 and 2002, reflecting discretion in the timing of their Medicare Rural Hospital Flexibility Programs. Conditional on availability, conversion to CAH status is a strategic hospital decision that may correlate with unobserved financial trajectories. We exploit this staggered rollout by comparing hospitals that convert to CAH status in early-adopting states to otherwise similar hospitals located in states where CAH designation was not yet available. Our primary control group therefore consists of “not-yet-treated” hospitals in slower-adopting states, rather than contemporaneous non-converters within the same state, whose non-conversion may reflect unobserved hospital-level selection.

We implement this comparison using a Synthetic Difference-in-Differences (SDID) estimator, which constructs weighted combinations of control hospitals to closely match treated hospitals’ pre-designation outcome trajectories. This approach is well suited to our setting, where treatment occurs in a small number of cohorts and conventional difference-in-differences comparisons may be sensitive to residual imbalance or differential trends across states.

Using hospital-level CAH designation data from the Flex Monitoring Team, annual survey data from the American Hospital Association, and financial data from the Hospital Cost Report Information System and IRS Form 990 filings, we estimate the effects of CAH designation on hospital finances, investment, capacity, and organizational outcomes. We find that CAH designation has no detectable effect on operating margins, with point estimates near zero across both estimators. Decomposing this null result into its components, however, reveals that net patient revenue per bed and total operating expenses per bed both decline by nearly identical amounts following designation---consistent with proportional financial contraction as hospitals downsize to satisfy eligibility requirements. Capital expenditures are imprecisely estimated and net fixed assets per bed change little. The primary response is an adjustment in scale: CAH designation is associated with reductions in total beds, obstetric capacity, and nursing staff, consistent with active resizing to meet program requirements. At the organizational level, CAH designation reduces hospital closures, with no meaningful effect on merger activity and a modest increase in system membership. Taken together, the results suggest that CAH designation preserves hospital survival largely by inducing continued operation at a smaller scale rather than by maintaining pre-designation capacity.

While CAH designation improves hospital survival, the implications for aggregate inpatient capacity are less clear. Reduced closures preserve capacity that would otherwise be lost, but hospitals that convert to CAH status also reduce beds and staffing to comply with eligibility requirements. To gauge the net effect, we combine our estimated impacts on closures and within-hospital capacity in a simple back-of-the-envelope calculation. Using our point estimates and a conversion rate consistent with observed CAH adoption during the initial rollout, the implied reduction in beds among converting hospitals outweighs the capacity preserved through avoided closures. These calculations are necessarily approximate and sensitive to assumptions about conversion intensity and closure risk, but they highlight an important mechanism: designation-based subsidies may preserve hospital presence primarily by encouraging continued operation at a smaller scale rather than by maintaining pre-designation capacity.

This paper contributes to three related literatures. First, a large body of work studies hospital closures and their consequences for access and patient outcomes \citep{buchmueller2006, holmes2006, lindrooth2003, hsia2011, chandra2023-nber31789, chatterji2024}. A complementary strand examines the determinants of hospital exit, emphasizing the role of financial distress, payer mix, and local market conditions \citep{sloan2003, harrison2007}. Second, a growing literature evaluates how policy-driven changes in hospital resources affect financial performance and survival. In the context of broad financing reforms, Medicaid expansions are associated with improved hospital margins and reductions in closure risk \citep{blavin2016, lindrooth2018, blavin2021}. Third, we relate to work studying Medicare’s targeted payment adjustments for low-volume or rural providers, which are often justified on access grounds but may change hospitals’ long-run incentives and scale of operation (e.g., through eligibility constraints and compliance requirements).

Our focus on CAH designation is distinct because it is a designation-based subsidy that simultaneously raises reimbursement and restricts scale, making its net implications for access theoretically ambiguous. Existing work has largely emphasized survival as the principal outcome, with less attention to how policies that prevent closure may simultaneously reshape hospitals’ operational scale and service capacity. Our analysis therefore complements \cite{carroll2025}, who studies CAH designation through simulated revenue changes and closure hazards, by estimating the reduced-form effects of designation itself on a broader set of outcomes—including capital investment, capacity, and staffing—while exploiting the staggered 1999–2002 rollout of state CAH availability. Our main SDID estimates focus on the 1999–2001 cohorts, which capture the core early adoption years of the CAH program. This broader perspective allows us to characterize not only whether hospitals remain open, but how the quantity and scope of care they provide changes when they do.

Our findings are particularly salient given renewed policy interest in designation-based approaches to hospital support. For example, the American Hospital Association has recently advocated for a new ``Metropolitan Anchor Hospital'' designation aimed at stabilizing urban safety-net hospitals with high Medicaid exposure. While such proposals differ in important ways from the CAH program, they share a common logic: providing targeted financial relief to hospitals deemed essential to local access. Our results suggest that designation-based subsidies can be effective at preserving hospital survival, but that they may do so by encouraging hospitals to operate at a smaller scale. As a result, policymakers considering similar interventions should weigh not only whether hospitals remain open, but also how designation affects the quantity and scope of care those hospitals ultimately provide.

\section{Institutional Background}
\label{sec:background}

The CAH program was introduced under the Balanced Budget Act of 1997 as part of a broader federal response to widespread rural hospital closures in the 1980s and early 1990s. The central goal was to preserve access to essential hospital services in sparsely populated areas by replacing inpatient prospective payments with cost-based reimbursements. Under the current program, eligible hospitals are reimbursed at 101\% of their Medicare-allowable costs, providing a guaranteed margin on Medicare patients and protecting hospitals against the risks inherent in the inpatient prospective payment system. By 2021, more than 1,300 hospitals had obtained CAH status, making it the most widely used federal designation to support rural hospital viability \citep{crs2023}.

Eligibility for CAH designation revolves around three criteria: 1) size; 2) rural location; and 3) proximity to other hospitals. In general, CAH-designated hospitals are small, rural, and relatively isolated; however, specific criteria have not necessarily been enforced rigidly over time. For example, hospitals in some otherwise urban Metropolitan Statistical Areas (MSAs) could seek urban-to-rural reclassification, further expanding CAH eligibility. Such a designation was made primarily on a case-by-case basis, generally governed by the extent to which the hospital was geographically isolated and served a rural population. Official requirements for urban-to-rural reclassification were adopted in the mid-2000s.\footnote{Eventually, hospitals seeking urban-to-rural reclassification could only be classified as rural provided they were located in a sub-county area with a Rural Urban Commuting Area code of 4 or more, but this criterion was not adopted until 2007} Similarly, CAH designation currently requires hospitals to be at least 35 miles from the nearest other hospital on ``primary roads'' or 15 miles on ``secondary roads,'' but prior to 2006, states could exempt hospitals from these proximity requirements as part of a ``necessary provider'' exception.\footnote{While the necessary provider exception was no longer allowed after 2006, those with such an exemption were grandfathered in and maintained their CAH status beyond 2006. When combining both rural location and distance criteria, an Office of the Inspector General report in 2013 found that ``Nearly two-thirds of CAHs would not meet the location requirements if required to re-enroll.''} Ultimately, while the CAH program is federally authorized, states play a key role in determining which hospitals may obtain CAH status via their Medicare Rural Hospital Flexibility Program (the “Flex Program”), which designates a state agency to review applications and determine CAH eligibility. Historically, eligibility criteria have therefore been more flexible in practice, enabling states to adapt the program to local hospital markets.

Other Medicare payment designations—such as low-volume hospital (LVH), Medicare dependent hospital (MDH), and sole community hospital (SCH)—serve related purposes but have been less consistently available. For example, the MDH and LVH programs expired in FY 2017 before being renewed through 2024. By contrast, the CAH designation has existed continuously since 1997, with precursors from state-level demonstration programs, and now accounts for over half of all such special designations \citep{crs2023}. The stability and breadth of CAH designation make it central to understanding federal policy toward rural hospitals, with implications for urban hospitals in light of recent advocacy for metropolitan anchor hospital designation.

Once designated, CAHs must maintain service and size restrictions, including treating no more than 25 inpatients per day, maintaining an average length of stay below 96 hours, and offering 24-hour emergency care. The tradeoff for receiving cost-based reimbursement is therefore reduced scope and operational flexibility, as hospitals must continuously satisfy these requirements to retain CAH status.

To organize the empirical predictions, consider a hospital serving publicly insured (Medicare) and commercially insured patients in quantities $(q_g, q_c)$ at per-unit prices $(p_g, p_c)$, with total operating costs $C(q_g + q_c)$. Let $\theta_{it} \in \{0,1\}$ indicate whether hospital $i$ holds CAH status in year $t$. CAH designation affects the hospital through two channels: a reimbursement premium $\alpha > 0$ on Medicare patients, and a capacity constraint that limits total patient volume. Period operating profits are
\begin{equation}
\pi = q_g \cdot p_g(1 + \alpha\theta) + q_c \cdot p_c - C(q_g + q_c), \quad \text{subject to } q_g + q_c \leq \bar{Q}(\theta),
\label{eq:cah_profit}
\end{equation}
where $\bar{Q}(1) < \bar{Q}(0)$ reflects CAH eligibility restrictions on inpatient census, length of stay, and service scope. The operating margin is $m = 1 - C/R$, where $R$ denotes total revenue.

The net effect on $m$ is ambiguous. Cost-based reimbursement raises per-unit Medicare revenue, which in isolation improves margins. But the binding capacity constraint forces reductions in total volume, contracting both revenue and costs. If hospitals downsize proportionally---reducing beds, staff, and services to satisfy the census cap---revenue and costs may fall in near-lockstep, leaving the margin ratio largely unchanged. The sign of the margin effect therefore depends on the relative magnitude of the per-unit reimbursement gain and the volume-driven contraction in the revenue base.

CAH designation may nonetheless affect organizational outcomes. A hospital exits when $\pi$ falls below a threshold $\underline{\pi}$ reflecting the opportunity cost of continued operation. Even if designation does not raise expected profits, cost-based reimbursement replaces variable prospective payments with a guaranteed cost-plus rate, reducing the variance of Medicare revenue and lowering the probability that $\pi < \underline{\pi}$ in any given period. This variance-reduction channel can decrease closure risk without improving average profitability. Mergers and system membership represent alternative organizational responses to financial pressure, with the incentive to affiliate depending on both the level and variability of standalone profits. These mechanisms guide our empirical analysis in Section~\ref{sec:design}.

\section{Data Sources and Key Variables}
\label{sec:data}

\subsection{Data Sources}

\paragraph{\textit{American Hospital Association (AHA) Annual Survey.}}  
The AHA survey provides a rich set of hospital characteristics for essentially all non-federal general acute care hospitals from 1980 onward, including information on ownership, size, teaching status, system membership, CAH designation, and detailed location information. The AHA surveys also include annual summaries of organizational changes in the ``Summary of Changes'' files, including hospital closures and mergers.

\paragraph{\textit{Flex Monitoring Team CAH Supplement.}}  
To track CAH status, we supplement the AHA surveys with data from the Flex Monitoring Team, which is ``a consortium of researchers from the Universities of Minnesota, North Carolina at Chapel Hill, and Southern Maine'' tasked with evaluating the Medicare Rural Hospital Flexibility Program. These data provide effective CAH designation dates, hospital locations, and capacity. Because the AHA’s critical access indicator is incomplete, we fuzzy-match the supplement to the AHA survey to recover accurate designation histories. This linkage allows us to observe the staggered rollout of CAH status beginning in the mid-1990s.\footnote{The official CAH program launched as part of the Balanced Budget Act of 1997; however, the CAH program was modeled after pre-existing demonstration programs such as the Rural Primary Care Hospital designation. As such, the Flex Monitoring Team identifies a handful of hospitals as CAHs dating back to 1994.}

\paragraph{\textit{IRS Form 990.}}  
Nonprofit hospitals are required to file Form 990 tax returns, which contain information on revenues, expenses, assets, and liabilities. These data allow us to construct operating margins and liquidity ratios. We link Form 990 filings to the AHA survey using Employer Identification Numbers (EINs) as well as fuzzy matching of names and locations.  

\paragraph{\textit{Healthcare Cost Report Information System (HCRIS).}}  
Medicare cost reports provide annual financial statements for virtually all facilities that participate in Medicare. We use these to construct operating margins and related measures analogous to those from the Form 990s. Since the cost reports are at the level of an individual hospital structure (rather than a system), we rely on financial measures constructed from HCRIS when possible.

\subsection{Key Variable Construction}

\paragraph{\textit{CAH designation.}}  
Our treatment variable is an indicator for whether a hospital is designated as a CAH in a given year. We construct this using both the AHA and the Flex Monitoring Team supplement, with the latter resolving discrepancies and providing precise effective dates. At the state level, we define treatment timing as the year of first designation in that state. Our construction of treated and control hospitals/states is presented in more detail in Section \ref{sec:design}.

\paragraph{\textit{Financial performance.}}
We construct six hospital-level financial outcomes using the HCRIS and Form 990 data.\footnote{Details of the construction of our financial measures are presented in the supplemental appendix.} First, we measure basic hospital profitability using the operating margin, defined as operating income divided by net patient revenue. This captures internal resources available to support ongoing operations. Second, we proxy short-run financial flexibility using the current ratio, defined as current assets divided by current liabilities, so that higher values indicate a greater ability to meet near-term obligations without external financing. Third, to decompose changes in operating margin into their component parts, we separately measure net patient revenue per bed and total operating expenses per bed, each deflated to constant 2010 dollars using the CPI-U and normalized by pre-period mean bed count. These component measures allow us to assess whether changes in margins reflect movements in revenue, costs, or both. We winsorize values at the 5th and 95th percentiles within each year and CAH-status cell and linearly interpolate remaining missing years within hospitals to smooth isolated reporting anomalies while preserving longer-run trends.

\paragraph{\textit{Capital stock and investment.}}
We measure capital intensity as real net fixed assets per bed, where net fixed assets include land, buildings, and equipment net of accumulated depreciation. We deflate this measure using the Consumer Price Index for All Urban Consumers (CPI-U) and normalize by the hospital's pre-period bed count. We further construct real capital expenditure per bed as a measure of the intensity of investment in long-term assets relative to hospital size. We approximate this using the year-to-year change in gross fixed assets (net plant plus accumulated depreciation) from HCRIS. When capital expenditures cannot be measured with HCRIS data, we construct analogous measures from Form 990 as the change in net fixed assets plus depreciation expense. We again winsorize values at the 5th and 95th percentiles within each year and CAH-status cell and linearly interpolate remaining missing years within hospitals to smooth isolated reporting anomalies while preserving longer-run trends.

\paragraph{\textit{Capacity and staffing.}}
We measure hospital capacity and staffing using three complementary indicators drawn from the AHA Annual Survey: total staffed beds, obstetric (OB) beds, and full-time equivalent (FTE) registered nurses. Total beds capture overall inpatient capacity, while OB beds provide a targeted measure of capacity in maternity services, a clinically and financially salient service line for many small and rural hospitals. FTE registered nurses proxy for labor inputs and operational capacity, reflecting hospitals' ability to staff inpatient services and sustain day-to-day clinical operations.

\paragraph{\textit{Organizational change.}}  
Closures and mergers are identified using the AHA’s summary of changes. We construct indicators for each event at the hospital-year level and aggregate to counts at the state-year level. To facilitate comparisons across states and over time, we rescale SDID estimates and plotted paths by the mean pre-period hospital count in treated states, reporting results as counts per 100 hospitals.  

\paragraph{\textit{Other covariates.}}  
From the AHA survey, we collect ownership, teaching status, system affiliation, and bed size. To capture geographic isolation, which is central to CAH eligibility, we compute the distance from each hospital to its nearest neighbor. We combine ZIP-code–based and latitude–longitude measures, taking the minimum of the two to approximate travel distance.  

\subsection{Descriptive Statistics}

Our hospital-level dataset covers 6,911 unique hospitals observed at any point from 1980–2019. We restrict to general, acute care, community hospitals, and we exclude Maryland, Alaska, Hawaii, and all territories (Puerto Rico, Guam, and others). The resulting panel contains consistent hospital identifiers, CAH designation histories, operating margins, and indicators of closures and mergers, along with covariates and geographic measures as discussed previously. 

Figure~\ref{fig:desc-panelview-treat} visualizes the staggered rollout of CAH designation at the state level. The white tiles in Figure~\ref{fig:desc-panelview-treat} denote years in which no hospital in that state had received CAH designation, and the gray tiles reflect years in which at least one hospital in that state had received CAH designation. From~\ref{fig:desc-panelview-treat}, we see CAH designations first arise as part of the early demonstration programs in 1994 and 1995, with 5 states having at least one hospital with such a designation by 1995. The program rapidly expanded once officially launched in 1999, with 20 states having at least one CAH in 1999, another 8 states in 2000, another 7 states in 2001, and 3 additional states in 2002. Figure~\ref{fig:desc-hosp-types} presents the analogous hospital-level data, reflecting the rise of CAH designation between 1999 and 2005, with the count of CAH-designated hospitals stabilizing near 1,300 hospitals by 2005.

We limit our empirical analysis to the years from 1995 through 2010 in order to align with the core years over which states adopted CAH designations (1999 through 2002) and avoid increasingly noisy estimates of operating margins in prior years. With this additional restriction to 1995-2010, Table~\ref{tab:summary-by-cah} provides descriptive statistics for CAHs and non-CAHs, focusing on the key CAH eligibility criteria of bed size, rural location, and proximity to other hospitals, as well as our primary outcome variables of operating margin, closures, and mergers. For CAHs, we present summary statistics separately for the years prior to CAH designation, the years with CAH designation, and all years combined (columns 1-3 of Table~\ref{tab:summary-by-cah}). Hospitals that received CAH designation after 2010 (before 1995) therefore enter into the ``pre'' (``post'') values but not vice versa, hence the different counts of pre and post within CAH designated hospitals. 

We highlight three points from Table \ref{tab:summary-by-cah}. First, CAHs are significantly smaller, more rural, and less proximate to other hospitals than are non-CAHs, consistent with the underlying CAH eligibility criteria. Second, demonstrating the relative flexibility in CAH eligibility criteria, there remains some overlap in the distribution of these characteristics between CAH and non-CAH. For example, among hospitals receiving CAH designation, the 90th percentile in bed size in pre-CAH years was 106. Meanwhile the 10th percentile of bed size for non-CAHs was 49. CAH designation therefore does not rigidly imply that hospitals are rural, smaller, or less proximate to others than non-CAHs, although on average this remains true. Third, we see that hospitals already with CAH designation are consistently smaller in terms of bed size than similar hospitals who eventually also receive CAH designation. For example, after receiving CAH designation, the mean CAH bed size and mean inpatient days per bed were 40 and 134, respectively. Meanwhile, the mean bed size and inpatient days per bed for CAHs before receiving designation were 54 and 165. The level changes in bed sizes and inpatient days per bed reflect the practical constraints placed on hospitals once receiving CAH designation, wherein such hospitals appear to reduce their number of beds and inpatient days in order to receive and maintain CAH designation.

\section{Effects of CAH Designation}
\label{sec:design}

Guided by our discussion in Section~\ref{sec:background} and surrounding Eq.~\eqref{eq:cah_profit}, we begin by estimating the reduced-form effects of CAH designation on hospital outcomes. We pursue a difference-in-differences identification strategy exploiting differential timing of CAH status and statewide CAH availability. Our estimation strategy employs a Synthetic Difference-in-Differences (SDID) framework to estimate the causal effect of CAH designation \citep{arkhangelsky2021}. We focus first on the effects of CAH on hospital finances, capacity, and investment, then turn to organizational outcomes including closures, mergers, and system membership.

\subsection{Empirical Strategy}
\label{sec:design-methods}

SDID estimates nonnegative unit weights on the comparison pool to reproduce the treated cohort’s pre-period trajectory and time weights to align pre-period averages, yielding a post-period ATT and treated/synthetic paths. We report cohort-level paths to make the identifying variation transparent and then aggregate across cohorts in event time by weighting each cohort’s path by its treated-unit count. Inference follows our implementation: jackknife standard errors.

Formally, let $Y_{it}$ denote the outcome for unit $i$ in period $t$, $D_{it}$ an indicator for treatment, and $G$ the set of treated units in a given cohort $c$ with pre-period $\mathcal{T}_0$ and post-period $\mathcal{T}_1$. The SDID estimator chooses unit weights $\omega=(\omega_j)_{j\in C}$ for comparison units $C$ and time weights $\lambda=(\lambda_t)_{t\in\mathcal{T}_0}$ to minimize
\begin{equation*}
\sum_{t\in\mathcal{T}_0}\Big( \bar Y_{Gt} - \sum_{j\in C}\omega_j Y_{jt}\Big)^2 
\quad \text{and} \quad
\bar Y_{G\mathcal{T}_0} - \sum_{t\in \mathcal{T}_0}\lambda_t\sum_{j\in C}\omega_j Y_{jt},
\end{equation*}
subject to $\omega_j\ge 0$, $\sum_{j\in C}\omega_j=1$, and $\lambda_t\ge 0$, $\sum_{t\in \mathcal{T}_0}\lambda_t=1$. The resulting synthetic controls reproduce the treated cohort’s pre-period trajectory and average level. The cohort-specific ATT is then
\begin{equation*}
\widehat{\tau}_c = \frac{1}{|\mathcal{T}_1|}\sum_{t\in \mathcal{T}_1}\Big(\bar Y_{Gt} - \sum_{j\in C}\omega_j Y_{jt}\Big).
\end{equation*}
We report both the path $\bar Y_{Gt}-\sum_{j\in C}\omega_j Y_{jt}$ over $t$ and the pooled ATT $\widehat\tau_c$. Cohort estimates are then aggregated in event time by weighting each cohort’s path by its treated-unit count. We summarize effects across cohorts using treated-unit–count weights for the pooled ATT and compute pooled standard errors using the same weighting scheme, treating cohort-specific estimates as approximately independent.

We prefer an SDID approach for two reasons. First, our analysis relies on a small number of states and hospitals receiving CAH designation, along with a limited pool of comparable control hospitals, for which SDID methods are less sensitive. Second, the criteria for CAH designation are not strictly mechanical in practice, which will tend to introduce unobserved heterogeneity and potential endogeneity into the treatment selection process. Hospitals that receive CAH designation may therefore systematically differ in a time-varying way from a simply-defined control group (e.g., small, rural hospitals in the same state). In assigning optimal weights to minimize the pre-treatment outcome differences, SDID methods intuitively create a control group for which the parallel trends assumption is more likely to be satisfied.\footnote{We complement our SDID estimates with the Callaway--Sant'Anna (CS) event-study estimator \citep{callaway2021}, which accommodates unbalanced panels and traces dynamic treatment effects. CS results are reported in the supplemental appendix.} 

In the remainder of this section, we discuss results by outcome category, beginning with financial performance and proceeding through investment, capacity, and organizational outcomes. A concise summary of the pooled estimates is presented in Table \ref{tab:sdid-overall}.

\subsection{Financial Performance}
\label{sec:financial}

We begin by examining how CAH designation affects hospital operating margins, the outcome most directly linked to the reimbursement change in Section~\ref{sec:background}. In this analysis, treatment is defined at the hospital level: a hospital is considered treated once it receives CAH designation, and the unit of observation is the hospital-year. The sample is restricted to facilities plausibly eligible for designation (i.e., small rural hospitals consistent with program criteria), excluding hospitals with missing financial data. This setup isolates the direct effect of adopting CAH status on hospital margins.

For each cohort, we construct a balanced panel of treated units and comparison hospitals over a symmetric window around the adoption year. For each treated cohort $i$, defined by the year of first CAH designation, we construct the treated group as hospitals that receive their initial CAH designation in year (i) and are observed over an event window $$([i-\text{pre},, i+\text{post}]).$$ The comparison pool combines two groups: (i) ``never-treated'' hospitals located in states that never adopt CAH designation during our sample, and (ii) ``not-yet-treated'' hospitals in later-adopting states that have no CAH hospitals as of the cohort year and do not convert within the event window. This design means that, for a 1999 CAH in Arkansas, for example, we can use non-CAH Kentucky hospitals through the event window as controls (Kentucky’s first CAH arrives in 2000), but we do not include as controls any hospitals in Georgia, where CAH was already available in 1999. The restriction on a state's first CAH designation is intended to reduce reliance on comparisons between early adopters and same-state or same-cohort non-adopters whose decision not to convert may reflect unobserved, time-varying hospital characteristics (e.g., management quality, lobbying success, or local politics) that also directly affect the outcomes we study. This restriction therefore pushes the design toward comparing treated hospitals to similar hospitals in ``slower'' Flex states that had not yet opened the CAH option in the cohort year, while still allowing those states to roll out CAH later and using their not-yet-treated hospitals as controls.\footnote{In the supplemental appendix, we show that our results are robust to varying lags for state-level CAH adoption. As the lag increases, potential bias from future treatment of the control units declines, along with our available control pool of hospitals. The resulting estimates are less precise but broadly consistent with our main specification and sample construction.}

Figure~\ref{fig:sdid-financial} reports SDID estimates of CAH designation on operating margin, the current ratio, and the components of operating margin---net patient revenue per bed and total operating expenses per bed. The pooled SDID ATT for operating margin is $-0.010$ (95\% CI $[-0.05, 0.02]$), centered near zero and statistically insignificant. The CS estimate is similarly null ($0.001$, 95\% CI $[-0.05, 0.05]$). As discussed in Section~\ref{sec:background}, the net effect on margins reflects the interaction of higher Medicare reimbursement with the operational restructuring induced by eligibility constraints. Decomposing the margin into its components clarifies this mechanism.

Panels (c) and (d) of Figure~\ref{fig:sdid-financial} show that both net patient revenue per bed and total operating expenses per bed decline following CAH designation, with remarkably similar magnitudes. The SDID estimates are $-203$ and $-208$ thousand 2010 dollars per bed, respectively, though both confidence intervals include zero. The CS estimates are larger and statistically significant: $-522$ $[-719, -325]$ for revenue and $-540$ $[-731, -348]$ for expenses. Importantly, the near-identical decline in revenue and expenses is consistent with proportional financial contraction---hospitals that downsize to satisfy CAH eligibility requirements experience parallel reductions in both financial inflows and outflows, leaving the ratio (and hence the operating margin) unchanged. We view this decomposition as suggestive rather than definitive, as the revenue and expense components exhibit differential pre-trends in some cohorts.\footnote{Total operating expenses per bed shows statistically significant differential pre-trends in all three cohorts, while net patient revenue per bed flags in two of three. Pre-period sensitivity analysis in the supplemental appendix shows that SDID point estimates are stable across pre-period lengths of 2--5 years, suggesting the results are not driven by differential trends.}

For the current ratio, the SDID and CS estimates diverge: the SDID pooled ATT is $-0.709$ (95\% CI $[-1.31, -0.11]$), while the CS estimate is $1.816$ (95\% CI $[1.30, 2.34]$). Given the small treated samples in financial outcomes (34--43 hospitals depending on the pre-period length) and the sensitivity of SDID to balanced-panel composition, we do not draw strong conclusions about the direction of liquidity effects.

\subsection{Capital Stock and Investment}
\label{sec:capital}
Figure~\ref{fig:sdid-capital} examines whether CAH designation translates into changes in capital intensity (net fixed assets per bed) and investment flows (capital expenditures per bed). The estimated effect on the installed capital stock is negative but modest and statistically indistinguishable from zero: the pooled ATT for net fixed assets per bed is $-0.099$ (95\% CI $[-0.27, 0.07]$). Estimated effects on capital expenditures are also negative but highly imprecise (ATT $= -1.246$, 95\% CI $[-5.81, 3.32]$). Taken together, these results do not support the conclusion that CAH designation meaningfully changes long-run capital intensity over the time horizon we study. Given the null margin results documented in Section~\ref{sec:financial}, this is unsurprising: the proportional contraction in revenue and expenses leaves little additional surplus to fund capital investment, while the reduction in beds and staffing may further reduce the demand for new capital.

\subsection{Capacity and Staffing}
\label{sec:capacity}
Figure~\ref{fig:sdid-capacity} shows that CAH designation is associated with a contraction in both capacity and labor inputs. Total staffed beds decline following adoption, with a pooled ATT of -4.081 beds (95\% CI [-7.70, -0.46]). The reduction is particularly salient in obstetric capacity: OB beds fall by -0.468 (95\% CI [-0.81, -0.12]). Staffing moves in the same direction, with FTE registered nurses declining by -2.023 (95\% CI [-3.44, -0.61]). These patterns are consistent with the program’s mechanical constraints (e.g., inpatient census limits and related operating requirements) and with hospitals actively resizing service capacity to obtain and maintain CAH status. Importantly, the staffing response tracks the scale response in beds, suggesting that CAH designation primarily induces a reduction in operational scale rather than a pure substitution from capital to labor or vice versa. 

\subsection{Organizational Changes}
\label{sec:orgchange}

Aggregating to the state-year level yields information on the number of hospitals, closures, mergers, and CAH adoptions in each state. Treatment timing is defined as the first observed CAH designation in a state, producing staggered adoption between 1999 and 2002. Event-time variables are defined relative to each state’s adoption year.

We next examine whether CAH designation altered hospitals' propensity to exit, consolidate, or join a larger system. Closures and mergers are defined at the state-year level as counts. To aid interpretation and comparability, we rescale SDID estimates and plotted paths by the mean pre-period hospital count in treated states, reporting results as counts per 100 hospitals. System membership is measured at the hospital-year level, so this outcome is estimated at the hospital level. Figure~\ref{fig:sdid-org-change} presents SDID results for (a) closures, (b) mergers, and (c) system membership. The synthetic controls track pre-designation trajectories reasonably well; the pre-period series is noisier for closures and mergers because these events are relatively rare.

The post-period estimates point to meaningful reductions in closures following CAH designation. The pooled ATT implies a decline of 0.483 closures per 100 hospitals (95\% CI $[-0.83, -0.13]$). Although the margin results in Section~\ref{sec:financial} show no improvement in average profitability, cost-based reimbursement may nonetheless reduce the variance of Medicare revenue, lowering the probability of acute financial distress that precipitates closure. By contrast, we find no evidence of a systematic effect on merger activity: the pooled ATT for mergers is 0.111 per 100 hospitals, with a confidence interval that includes zero (95\% CI $[-0.19, 0.41]$). At the hospital level, we find a modest increase in system membership (ATT $= 0.074$, 95\% CI $[0.01, 0.14]$), suggesting some movement toward affiliation even as closures decline. Taken together, the organizational-change results indicate that CAH designation primarily operates on the extensive margin of survival rather than affecting consolidation.


\subsection{Robustness: Alternative Control Group}
\label{sec:robustness_elig}

Our main identification strategy relies on the staggered timing of state-level CAH program adoption to construct control groups of not-yet-treated hospitals. While this design limits contamination from within-state selection, it restricts both the number of treated cohorts (1999--2001) and the control pool. To assess the sensitivity of our results to this particular control group construction, we implement an alternative eligibility-restricted design that relaxes the state-timing requirement entirely.

In the eligibility-restricted design, the treated group consists of hospitals that convert to CAH status in a given year, and the control group includes all non-CAH hospitals with 50 or fewer beds---both those that never convert and those that have not yet converted---regardless of whether their state has adopted the CAH program. This design widens both the cohort range (1999--2005, yielding seven cohorts rather than three) and the treated sample (89--237 hospitals per cohort, compared to 7--19 in the state-timing design), while using approximately 989 non-CAH small hospitals as controls. The tradeoff is that the control group may include hospitals that chose not to convert despite eligibility, reintroducing potential selection concerns that the state-timing design was intended to mitigate. Nevertheless, the design provides a useful check on whether our main estimates are driven by the specific control group construction or by the state-timing restriction itself.

Figure~\ref{fig:forest-sdid} summarizes the comparison, plotting SDID ATT estimates and 95\% confidence intervals from the state-timing and eligibility-restricted designs side by side for each outcome. The capacity effects---total beds, OB beds, and FTE RNs---are robust across both designs and often larger in magnitude in the eligibility-restricted specification, consistent with later-adopting cohorts (2002--2005) experiencing similar or stronger downsizing. Financial results are qualitatively similar: operating margin is null under both designs, and revenue and expense point estimates are near zero in the eligibility-restricted specification (compared to negative but imprecise in the state-timing design). System membership and the current ratio are consistent in sign, with the current ratio divergence between SDID and CS estimates substantially attenuated in the eligibility-restricted design ($-0.121$ versus $-0.709$ in the state-timing design). Full results from the eligibility-restricted design are reported in the supplemental appendix.


\subsection{Heterogeneity}
\label{sec:heterogeneity}

We examine heterogeneity in treatment effects along four dimensions: hospital ownership (government versus nonprofit), geographic isolation (distance to nearest hospital above versus below the treated-hospital median), pre-treatment financial health (operating margin above versus below the treated-hospital median, measured over 1996--1998), and pre-treatment system membership. For each dimension, we split both treated and control hospitals and estimate SDID separately for each subgroup using the eligibility-restricted design, which provides greater statistical power for subgroup analysis with seven cohorts and a larger treated sample. Figure~\ref{fig:het-forest} summarizes the results.

Capacity reductions following CAH designation are pervasive across all subgroups, but their magnitude varies meaningfully. Nonprofit hospitals exhibit larger reductions in beds ($-9.1$ versus $-5.8$ for government hospitals), OB beds ($-0.92$ versus $-0.39$), and FTE RNs ($-4.6$ versus $-2.2$), consistent with government hospitals facing political or institutional constraints that limit downsizing even when program eligibility rules incentivize reducing bed counts. Hospitals with below-median pre-treatment margins experience larger reductions in FTE RNs ($-5.2$ versus $-1.3$) and inpatient days per bed ($-27.5$ versus $-19.0$), suggesting that the operational contraction induced by CAH designation is more pronounced among financially weaker hospitals.

Geographic isolation shapes organizational rather than capacity outcomes. Isolated and proximate hospitals both reduce beds and staffing at comparable magnitudes, but only proximate hospitals show a significant increase in system membership post-CAH (ATT $= 0.076$, $p < 0.05$), while isolated hospitals show no significant change. This pattern is consistent with system membership being more feasible for hospitals with nearby potential partners. The system membership dimension tells a complementary story: hospitals already in a system deepen those ties post-designation (ATT $= 0.074$, significant), while independent hospitals show a small and insignificant decrease ($-0.054$). Financial outcomes remain imprecise across all subgroups, consistent with the main results.


\subsection{Net Capacity Effects}
\label{sec:net_capacity}

Our estimates imply two offsetting margins of adjustment for inpatient capacity. CAH designation reduces hospital closures, preserving capacity that would otherwise disappear, but it also induces within-hospital resizing among converting facilities. We quantify this tradeoff with a simple accounting exercise.

Consider a group of $N$ hospitals. Let $\Delta B$ denote the estimated change in total beds \emph{per converting (CAH) hospital}, and let $\Delta C$ denote the estimated change in closures \emph{per 100 hospitals}. Let $\bar B_{\text{close}}$ denote the average number of beds among the marginal hospitals that would otherwise close absent CAH designation. If a share $\rho$ of hospitals convert to CAH status over the relevant post period, then the implied net change in total beds is
\begin{equation}
\Delta \text{Beds}^{net}
=
N\left(\frac{-\Delta C}{100}\right)\bar B_{\text{close}}
+
N\rho\,\Delta B,
\label{eq:net_beds}
\end{equation}
where the first term captures beds preserved via avoided closures and the second term captures beds lost (or gained) through within-hospital changes among CAH converters.

We implement \eqref{eq:net_beds} using our pooled SDID estimates: $\Delta C=-0.483$ closures per 100 hospitals and $\Delta B=-4.081$ beds per converting hospital. For a back-of-the-envelope conversion intensity we set $\rho=0.30$, consistent with observed adoption rates among hospitals with $\leq 50$ beds in states that made CAH designation available by 2005. To proxy $\bar B_{\text{close}}$, we use the mean pre-CAH bed size from Table~\ref{tab:summary-by-cah}, $\bar B_{\text{close}}\approx 54$. Dividing \eqref{eq:net_beds} by $N$ yields the net effect per hospital:
\begin{equation*}
\frac{\Delta \text{Beds}^{net}}{N}
=
\left(\frac{0.483}{100}\right)\times 54
+
0.30\times(-4.081)
\approx
0.261 - 1.224
=
-0.964.
\end{equation*}
Thus, under these assumptions, the point estimates imply a net reduction of approximately $0.96$ beds per hospital (or $0.96N$ beds for a group of $N$ hospitals). Intuitively, the estimated reduction in closures preserves some capacity, but the within-hospital bed reductions among converting hospitals dominate at the observed conversion rates.

Because \eqref{eq:net_beds} is linear in $\rho$, it is straightforward to compute the break-even conversion share $\rho^{*}$ at which preserved beds from avoided closures exactly offset resizing among converters. Setting $\Delta \text{Beds}^{net}=0$ and solving for $\rho$ gives
\begin{equation}
\rho^{*}
=
\frac{\left(\frac{-\Delta C}{100}\right)\bar B_{\text{close}}}{-\Delta B}.
\label{eq:rho_star_beds}
\end{equation}
Using $\Delta C=-0.483$, $\Delta B=-4.081$, and $\bar B_{\text{close}}\approx 54$ implies
$$
\rho^{*}
=
\frac{(0.483/100)\times 54}{4.081}
\approx
0.064.
$$
That is, about 6\% of hospitals would need to convert for the point estimates to imply break-even bed capacity. When $\rho$ exceeds this threshold---as in our setting once the program is broadly available---the implied net effect on total beds is negative, reflecting that CAH designation preserves hospital presence primarily through continued operation at a smaller scale rather than through maintenance of pre-designation capacity.

\paragraph{\textit{Accounting for utilization.}}
The bed-based calculation in \eqref{eq:net_beds} captures the extensive margin of capacity but ignores changes in how intensively beds are used. Our estimates indicate that CAH designation also reduces inpatient days per bed by approximately $6.7$ days, though this effect is imprecisely estimated. Converting hospitals may not only have fewer beds but also operate those beds at lower intensity. To incorporate this margin, we extend the accounting exercise to total inpatient days.

Let $\bar U_0$ denote pre-CAH inpatient days per bed, and let $\Delta U$ denote the estimated change in utilization per bed. Using values from Table~\ref{tab:summary-by-cah}, $\bar U_0 \approx 165$ days per bed. The total change in inpatient days per converting hospital combines bed and utilization effects:
$$
\Delta D = \Delta B \cdot \bar U_0 + \bar B_{\text{close}} \cdot \Delta U + \Delta B \cdot \Delta U.
$$
Substituting our estimates yields
$$
\Delta D = (-4.081)(165) + (54)(-6.663) + (-4.081)(-6.663) \approx -673 - 360 + 27 = -1{,}006
$$
inpatient days per converting hospital. Avoided closures preserve $\bar B_{\text{close}} \times \bar U_0 = 54 \times 165 = 8{,}910$ inpatient days per marginal hospital. The net effect on inpatient days per hospital is therefore
$$
\frac{\Delta D^{net}}{N}
=
\left(\frac{0.483}{100}\right) \times 8{,}910
+
0.30 \times (-1{,}006)
\approx
43.0 - 301.8
=
-258.8.
$$
The break-even conversion share under this measure is $\rho^{*} = 43.0 / 1{,}006 \approx 0.043$, or about 4\%. Accounting for utilization thus implies a less favorable capacity tradeoff: CAH designation must reduce closures by more, or fewer hospitals must convert, to avoid net reductions in service capacity.

\paragraph{\textit{Sensitivity to conversion and closure effects.}}
The net-capacity calculations depend primarily on two quantities: the share of hospitals that convert to CAH status, $\rho$, and the magnitude of the closure response, $\Delta C$. Because the expressions are linear in both parameters, it is straightforward to assess how the implied net effect would change under alternative scenarios. Holding fixed the estimated within-hospital effects, an increase in $\rho$ mechanically amplifies the capacity loss from downsizing, while a larger (more negative) value of $\Delta C$ increases the amount of capacity preserved through avoided closures.

The marginal effect of increasing the conversion share on net beds is $\partial(\Delta \text{Beds}^{net}/N)/\partial\rho=\Delta B \approx -4.1$ beds per ten-percentage-point increase in conversion. By contrast, the marginal effect of strengthening the closure response is $\partial(\Delta \text{Beds}^{net}/N)/\partial(-\Delta C)=\bar B_{\text{close}}/100 \approx 0.54$ beds per hospital. Thus, increases in conversion intensity must be accompanied by disproportionately large reductions in closure risk in order to offset the resulting downsizing.

We view these calculations as suggestive rather than definitive. They underscore that, unless CAH designation substantially reduces the probability of closure beyond what we estimate, higher rates of conversion will tend to imply larger net reductions in inpatient capacity, even as the program succeeds in preserving hospital presence.




\section{Discussion}
\label{sec:conclusions}

This paper examines the effects of Critical Access Hospital designation on hospital finances, capacity, staffing, and organizational outcomes. Our central finding is that CAH designation achieves its primary goal of reducing hospital closures, but it does so at the cost of operational scale: converting hospitals shed beds, obstetric capacity, and nursing staff, consistent with the program's eligibility constraints. Operating margins are unaffected, not because nothing changes, but because revenue and expenses contract in near-lockstep as hospitals downsize. The result is a ``survival versus scale'' tradeoff in which designation preserves hospital presence primarily by encouraging continued operation at a smaller scale rather than by maintaining pre-designation capacity.

Our primary identification strategy exploits the staggered rollout of state-level CAH availability, comparing hospitals that convert in early-adopting states to otherwise similar hospitals in states where the program was not yet available. This design limits reliance on within-state comparisons between converters and contemporaneous non-converters, whose decision not to convert may reflect unobserved hospital-level selection. However, it cannot fully eliminate the possibility that hospitals selecting into CAH status differ from controls along unobservable dimensions. The alternative control group analysis in Section~\ref{sec:robustness_elig} provides reassurance: relaxing the state-timing restriction and widening the control pool to all non-CAH hospitals with 50 or fewer beds yields qualitatively similar results, with capacity effects that are often larger in magnitude. The heterogeneity analysis in Section~\ref{sec:heterogeneity} further demonstrates that the capacity reductions are pervasive across hospital ownership types, levels of geographic isolation, and pre-treatment financial health, reinforcing the interpretation that these effects reflect the program's design rather than selection by a particular subset of hospitals.

Several limitations warrant discussion. While capacity and organizational outcomes generally satisfy standard pre-trend diagnostics, the revenue and expense decomposition exhibits differential pre-trends in some cohorts. We present this decomposition as suggestive rather than definitive evidence; the stability of SDID point estimates across pre-period lengths of 2--5 years provides some reassurance, but we cannot rule out the possibility that pre-existing differences between treated and control hospitals contribute to the estimated effects. The null margin result itself does not rely on the decomposition and is robust across estimators, control group constructions, and pre-period lengths. Financial outcomes more broadly rely on balanced panels of 34--43 treated hospitals because HCRIS coverage ramps up gradually through the late 1990s and the SDID estimator requires each unit to be observed in every period. We report cohort-specific estimates and pre-period sensitivity analyses throughout to allow readers to assess the stability of the aggregate results. The current ratio is the one outcome for which SDID and Callaway--Sant'Anna produce qualitatively different results; this divergence is substantially attenuated in the eligibility-restricted design, consistent with balanced-panel composition effects, and we do not draw strong conclusions about the direction of liquidity effects.

CAH designation preserves hospital \textit{presence} but not pre-designation \textit{capacity}, and this distinction has direct implications for evaluating the program's effectiveness. As the back-of-the-envelope calculations in Section~\ref{sec:net_capacity} demonstrate, the break-even conversion rate is approximately 6\% for beds and 4\% for inpatient days---well below observed conversion rates in states that adopted the program. These findings are relevant to ongoing debates about designation-based hospital support. The Rural Emergency Hospital (REH) designation, introduced in 2023, represents a further point along the survival-scale tradeoff: REH status eliminates inpatient services entirely in exchange for enhanced outpatient reimbursement. Our results for CAH---where the capacity constraints are less severe---suggest that even modest eligibility restrictions can generate meaningful reductions in service scope. Whether the access value of maintaining a hospital presence outweighs the diminished inpatient capacity is ultimately a welfare question that our analysis cannot resolve without patient-level data on utilization, health outcomes, and travel burden.

Several extensions would sharpen the analysis. Alternative estimators designed for unbalanced panels, such as the generalized synthetic control method \citep{xu2017}, would allow the full sample to contribute to financial outcome estimates without the attrition imposed by balanced-panel requirements. Downstream analyses linking CAH designation to patient-level outcomes---including emergency department utilization, inpatient mortality, and travel distances---would directly inform the welfare implications of the survival-scale tradeoff documented here.

\pagebreak
\bibliography{BibTeX_Library}


%%% TABLES AND FIGURES

\clearpage
\newpage
\section*{Tables and Figures}


%% Descriptive Stats
\begin{figure}[htb]
\centering
\begin{minipage}[h]{6in}
\caption{Staggered rollout of CAH designation by state.}
\centerline{\includegraphics[scale=0.5]{../results/desc-panelview-treat.png}}
\label{fig:desc-panelview-treat}
\end{minipage}
\end{figure}

\clearpage
\newpage

\begin{figure}[htb]
\centering
\begin{minipage}[h]{6in}
\caption{Counts of CAH and non-CAH hospitals, 1980–2019.}
\centerline{\includegraphics[scale=0.5]{../results/desc-hosp-types.png}}
\label{fig:desc-hosp-types}
\end{minipage}
\end{figure}

\clearpage
\newpage

\begin{table}
\centering
\footnotesize
\begin{minipage}[h]{6in}
\caption[caption]{\textbf{Summary Statistics by CAH Designation}\footnote{Summary statistics for CAH and non-CAH hospitals, with CAHs further divided into years prior to designation, after designation, and all years. Numbers reflect means or counts, and brackets denote the 10th and 90th percentile ranges. Data are limited to general, acute care, community hospitals over the years 1995 through 2010. Distance reflects miles to the nearest neighboring hospital, as calculated as discussed in Section \ref{sec:data}; bed size is from the AHA variable ``BDTOT''; rural status is based on an indicator for whether the hospital is located in a rural CBSA as per the AHA survey; inpatient days is from the AHA variable ``IPDTOT'' and measured per bed; closures and mergers are identified from the AHA summary of changes files; operating margin is total revenue less total expenses, divided by total revenue; current ratio is current assets divided by current liabilities; and system membership indicates whether the hospital belongs to a multi-hospital system. }}
\centerline{%
    \begin{table}[!h]
\centering
\caption{Hospital characteristics by CAH status}
\centering
\begin{tabular}[t]{lcccc}
\toprule
\multicolumn{1}{c}{ } & \multicolumn{3}{c}{CAHs} & \multicolumn{1}{c}{ } \\
\cmidrule(l{3pt}r{3pt}){2-4}
 & Pre & Post & Ever CAH & Never CAH\\
\midrule
Rural (%) & 66.21 & 69.8 & 67.84 & 9.04\\
Bed size & 55.66 & 39.66 & 48.4 & 218.46\\
 & {}[21.00, 103.00] & {}[15.00, 83.00] & {}[18.00, 95.00] & {}[49.00, 448.00]\\
Distance to nearest neighbor & 18.53 & 19.34 & 18.9 & 7.01\\
 & {}[2.51, 32.25] & {}[8.74, 31.85] & {}[6.90, 32.05] & {}[0.00, 18.92]\\
\addlinespace
Inpatient days & 10.91 & 8.1 & 9.64 & 51.93\\
 & {}[1.82, 26.20] & {}[1.12, 21.53] & {}[1.42, 24.30] & {}[7.53, 114.79]\\
Margin & -0.06 & -0.06 & -0.06 & -0.01\\
 & {}[-0.25, 0.07] & {}[-0.24, 0.07] & {}[-0.25, 0.07] & {}[-0.15, 0.13]\\
Closures & 8 & 11 & 19 & 273\\
\addlinespace
Mergers & 7 & 1 & 8 & 392\\
Hospitals & 1356 & 1306 & 1388 & 4105\\
Observations & 11616 & 9640 & 21256 & 51463\\
\bottomrule
\end{tabular}
\end{table}

}
\label{tab:summary-by-cah}
\end{minipage}
\end{table}

%% Financial Performance
\clearpage
\newpage
\begin{figure}[htb]
\centering
\begin{minipage}[h]{6in}
\caption[caption]{SDID Estimates of CAH Status on Hospital Financial Performance\footnote{Results from synthetic difference-in-differences (DD) on the estimated effect of CAH status on hospital financial outcomes. Panel (a) presents results for hospital operating margin, panel (b) for current ratio, panel (c) for net patient revenue per bed, and panel (d) for total operating expenses per bed. Revenue and expense measures are deflated by the CPI-U and expressed in thousands of 2010 dollars per bed, with mean hospital beds calculated from 1995 up to (not including) the cohort year. The figures present the mean outcome among treated and synthetic control hospitals over time, with the synthetic DD treatment effect estimates noted in the upper right of each figure. We provide estimates of the average treatment effect on the treated (ATT) for each cohort (1999, 2000, and 2001) as well as a weighted average overall ATT, as detailed in the main text.}}
\begin{tabular}{cc}
\includegraphics[height=4in,width=2.7in,keepaspectratio]{../results/margin-sdid.png} & \includegraphics[height=4in,width=2.7in,keepaspectratio]{../results/currentratio-sdid.png} \\
\small (a) Operating Margin & \small (b) Current Ratio \\
\includegraphics[height=4in,width=2.7in,keepaspectratio]{../results/netpatrev-sdid.png} & \includegraphics[height=4in,width=2.7in,keepaspectratio]{../results/totopexp-sdid.png} \\
\small (c) Net Patient Revenue per Bed & \small (d) Operating Expenses per Bed
\end{tabular}
\label{fig:sdid-financial}
\end{minipage}
\end{figure}


%% Capital Stock and Investment
\clearpage
\newpage
\begin{figure}[htb]
\centering
\begin{minipage}[h]{6in}
\caption[caption]{SDID Estimates of CAH Status on Hospital Capital Stock and Investment\footnote{Results from synthetic difference-in-differences (DD) on the estimated effect of CAH status on hospital capital stock and investment measures. Panel (a) presents results for net fixed assets and panel (b) presents results for capital expenditures. Both measures are deflated by the CPI-U and expressed per bed, with mean hospital beds calculated from 1995 up to (not including) the cohort year. The figures present the mean outcome among treated and synthetic control hospitals over time, with the synthetic DD treatment effect estimates noted in the upper right of each figure. We provide estimates of the average treatment effect on the treated (ATT) for each cohort (1999, 2000, and 2001) as well as a weighted average overall ATT, as detailed in the main text.}}
\begin{tabular}{cc}
\includegraphics[height=4in,width=2.7in,keepaspectratio]{../results/netfixed-sdid.png} & \includegraphics[height=4in,width=2.7in,keepaspectratio]{../results/capex-sdid.png} \\
\small (a) Net Fixed Assets & \small (b) Capital Expenditures (CapEx)
\end{tabular}
\label{fig:sdid-capital}
\end{minipage}
\end{figure}

%% Capital Stock and Investment
\clearpage
\newpage
\begin{figure}[htb]
\centering
\begin{minipage}[h]{6in}
\caption[caption]{SDID Estimates of CAH Status on Hospital Capacity and Staffing\footnote{Results from synthetic difference-in-differences (DD) on the estimated effect of CAH status on hospital capacity and staffing measures. Panel (a) presents results for total hospital beds, panel (b) presents results for obstetric beds, panel (c) presents results for full time equivalent (FTE) registered nurses (RNs), and panel (d) presents results for total inpatient days per hospital bed. The figures present the mean outcome among treated and synthetic control hospitals over time, with the synthetic DD treatment effect estimates noted in the upper right of each figure. We provide estimates of the average treatment effect on the treated (ATT) for each cohort (1999, 2000, and 2001) as well as a weighted average overall ATT, as detailed in the main text.}}
\begin{tabular}{cc}
\includegraphics[height=4in,width=2.7in,keepaspectratio]{../results/beds-sdid.png} & \includegraphics[height=4in,width=2.7in,keepaspectratio]{../results/beds_ob-sdid.png} \\
\small (a) Total Beds & \small (b) Obstetric Beds \\
\includegraphics[height=4in,width=2.7in,keepaspectratio]{../results/ftern-sdid.png} &  \includegraphics[height=4in,width=2.7in,keepaspectratio]{../results/ipdays-sdid.png} \\
\small (c) Full Time Equivalent RNs & \small (d) Inpatient Days per Bed
\end{tabular}
\label{fig:sdid-capacity}
\end{minipage}
\end{figure}


%% Closures and Mergers
\clearpage
\newpage
\begin{figure}[htb]
\centering
\begin{minipage}[h]{6in}
\caption[caption]{SDID Estimates of CAH Status on Organizational Changes\footnote{Results from synthetic difference-in-differences (DD) on the estimated effect of CAH status on organizational outcomes. Panels (a) and (b) present state-level closures and mergers (counts rescaled by the mean pre-period hospital count to express outcomes per 100 hospitals). Panel (c) presents hospital-level system membership. The figures present the mean outcomes among treated units and synthetic controls, with the synthetic DD treatment effect estimates noted in the upper right of each figure. We provide estimates of the average treatment effect on the treated (ATT) for each cohort (1999, 2000, and 2001) as well as a weighted average overall ATT, as detailed in the main text.}}
\begin{tabular}{cc}
\includegraphics[height=4in,width=2.7in,keepaspectratio]{../results/closure-rate-sdid.png} & \includegraphics[height=4in,width=2.7in,keepaspectratio]{../results/merger-rate-sdid.png} \\
\small (a) Closures (per 100 hospitals) & \small (b) Mergers (per 100 hospitals) \\
\includegraphics[height=4in,width=2.7in,keepaspectratio]{../results/system-sdid.png} & \\
\small (c) System Membership &
\end{tabular}
\label{fig:sdid-org-change}
\end{minipage}
\end{figure}

\clearpage
\newpage

\begin{figure}[htb]
\centering
\begin{minipage}[h]{6in}
\caption[caption]{SDID Estimates across Control Group Constructions\footnote{Forest plot comparing SDID ATT estimates from the state-timing control group (filled circles) and the eligibility-restricted control group (open circles). Each row corresponds to a single outcome, with point estimates and 95\% confidence intervals. The eligibility-restricted design uses all non-CAH or not-yet-CAH hospitals with $\leq 50$ beds as controls regardless of state-level adoption timing, with cohorts spanning 1999--2005. Closures and mergers are estimated at the state level and are available only for the state-timing design.}}
\centerline{\includegraphics[width=5.5in,height=8in,keepaspectratio]{../results/forest-sdid.png}}
\label{fig:forest-sdid}
\end{minipage}
\end{figure}

\clearpage
\newpage

\begin{figure}[htb]
\centering
\begin{minipage}[h]{6in}
\caption[caption]{Heterogeneity in SDID Treatment Effects\footnote{Forest plots of subgroup-specific SDID ATT estimates from the eligibility-restricted design (cohorts 1999--2005). Each panel displays a different heterogeneity dimension; within each panel, rows correspond to outcomes and points distinguish the two subgroups. Horizontal bars denote 95\% confidence intervals. Ownership splits government versus nonprofit hospitals (excluding for-profits). Geographic isolation splits hospitals above versus below the treated-hospital median distance to the nearest hospital. Pre-treatment margin splits hospitals above versus below the treated-hospital median operating margin over 1996--1998. System membership splits hospitals that were versus were not members of a multi-hospital system prior to CAH designation.}}
\centerline{\includegraphics[width=\textwidth]{../results/het-forest-combined.png}}
\label{fig:het-forest}
\end{minipage}
\end{figure}

\clearpage
\newpage

\begin{table}
\centering
\footnotesize
\begin{minipage}[h]{6in}
\caption[caption]{\textbf{Summary of SDID ATTs}\footnote{Summary of Average Treatment Effects on the Treated (ATTs) from our synthetic difference-in-differences estimation. The table presents the weighted average overall ATT, as detailed in the main text, along with 95\% confidence intervals. $N_{tr}$ reports the total number of treated units across cohorts in the SDID balanced panel. Callaway--Sant'Anna event-study estimates are reported in the supplemental appendix.}}
\centerline{%
    Outcome & SDID ATT & SDID 95\% CI & CS ATT & CS 95\% CI \\
Operating margin & 0.035 & [0.00, 0.07] & 0.006 & [-0.04, 0.05] \\
Current ratio & -0.020 & [-0.91, 0.87] & 1.363 & [0.80, 1.93] \\
Net fixed assets & -0.090 & [-0.27, 0.09] & -0.144 & [-0.27, -0.01] \\
Capital expenditures per bed & -0.906 & [-5.37, 3.56] & 0.694 & [-3.28, 4.67] \\
Total beds & -4.081 & [-7.70, -0.46] & -16.779 & [-30.18, -3.38] \\
OB beds & -0.468 & [-0.81, -0.12] & -4.280 & [-6.03, -2.53] \\
FTE RNs & -2.023 & [-3.44, -0.61] & -60.022 & [-72.82, -47.23] \\
Inpatient days per bed & -6.663 & [-18.49, 5.16] & -83.638 & [-118.11, -49.16] \\
System membership & 0.074 & [0.01, 0.14] & 0.145 & [0.02, 0.27] \\
Closures & -0.483 & [-0.83, -0.13] & -0.362 & [-0.76, 0.03] \\
Mergers & 0.111 & [-0.19, 0.41] & -0.118 & [-0.63, 0.39] \\

}
\label{tab:sdid-overall}
\end{minipage}
\end{table}


\end{document}
